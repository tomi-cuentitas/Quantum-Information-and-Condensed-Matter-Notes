\iffalse
%\subsection{Ideal Bose Gas and Bose-Einstein Condensation}

The grand partition function of an ideal bosonic gas reads 

\[
    \mathcal{Z}_G = \prod_{\alpha} \bigg(
        1 - e^{-\beta (\epsilon_{\alpha}-\mu)}
    \bigg)^{-1}, \qquad
    \mathcal{Z}_G = \prod_{{\bf k}} \bigg(
        1 - e^{-\beta (\epsilon_{{\bf k}}-\mu)}
    \bigg)^{-1}
\]

where $\alpha$ labels the quantum states of the gas, and where ${\bf k}$ is the momentum associated to a plane-wave expansion of the partition function. 

It is clear that, at finite temperatures, $\mu < \varepsilon_{{\bf k}}$, for all momenta ${\bf k}$, otherwise, the occupation number of some $k'$-state, with $\varepsilon_{k'} = \mu$, will diverge. 
Since the lowest-energy state has energy $\varepsilon_0$, it follows that $\mu \leq 0$.
However, at zero temperature, this condition can be relaxed. \medbreak

Since bosonic states have no restriction on their occupation numbers, special care must be taken when making transitions from a summation to an integral over ${\bf k}$. In effect, contrast the following two cases.\medbreak

Consider a 3D system in contact with a particle reservoir that fixes the value of $\mu$. 
One finds the equation of state \ref{el2020advanced} 

\[
    PV = \frac{2}{3} U.
\]

On the contrary, if now the system of bosonic particles is enclosed in an impermeable box and in contact with a temperature reservoir $T$, with particle number $N$ fixed, the following equation of motion is obtained 

\[
    \frac{N \lambda_T^3}{g\Omega} = \frac{n \lambda_T^3}{g} = F_{\frac{3}{2}}(\alpha), 
\]

where $\lambda_T = \sqrt{\frac{2\pi \hbar^2}{m k_B T}}$ is the thermal de Broglie wavelength and where $F_{\frac{3}{2}}$ is the polylogarithmic function. 

If the temperature is lowered to $T' < T$, then $\lambda_{T'} > \lambda_T$, and hence $\alpha' = e^{\beta' \mu} > \alpha$, since $\mu < 0$. Moreover, $\mu$ has 
\fi

\subsection{Interacting Bosons and Superfluidity}

\blanky \bigbreak 
\begin{tcolorbox}[colback = yellow, title = Physical Context]

Before the experimental observation of Bose-Einstein condenstation in optically trapped super-cooled atoms, it was thought to occur in superfluid He$^4$

\end{tcolorbox}. 

%\subsection{}

\clearpage

\subsection{Bogoliubov Theory of Superfluid Helium: Weakly interacting Bosons}

Current understanding of BEC as the microscopic basis of Landau's theory of superfluidty was first derived by Bogoliubov in 1947.
Bogoliubov's theory is based on a physical assumption that a weakly-non-ideal Bose gas can undergo a condensation similar to the traditional BEC. 
The existence of the Bose condensate leads to a unique macroscopic wavefunction of the whole system, e.g. a collective effect.
Hence, weak interactions transform single-particle excitations into a spectrum of collective excitations. 

The Hamiltonian, in field operators representations, reads as follows 

\begin{equation}
    {\bf H} = \int_{\R^3} d{\bf } \blanky \nabla \psi^\dagger_{\bf x} \nabla \psi_{\bf x} + \frac{1}{2} \int_{\R^3} d{\bf x'} V({\bf x'-x}) \psi^\dagger_{{\bf x}} \psi^\dagger_{{\bf x'}} \psi_{{\bf x}} \psi_{{\bf x'}},
\end{equation}

where $V(\bf x)$ is a short range two-particle interaction potential. 
In the particular case of a uniform gas in volume ${\Omega}$, the field operators can be expanded as 

\begin{equation}
    \psi = \sum_{{\bf k} \in \Lambda} \frac{e^{i {\bf k} \cdot {\bf x}}}{\sqrt{\Omega}} {\bf b}_{{\bf k}},
\end{equation}

where the ${\bf b}_{{\bf k}}$ are annihilation operators.
Then, the Hamiltonian reads 

\begin{equation}
    {\bf H} = \sum_{{\bf k} \in \Lambda} \varepsilon_{{\bf k}} {\bf b}_{{\bf k}}^\dagger {\bf b}_{{\bf k}} + \frac{1}{2\Omega} \sum_{{\bf k, k',q}} V_{{\bf q}} {\bf b}_{{\bf k-q}}^\dagger {\bf b}_{{\bf k'+q}}^\dagger {\bf b}_{{\bf k}'} {\bf b}_{{\bf k}},
\end{equation}

where $\varepsilon_{{\bf k}}$ is the energy of a helium atom of mass $M$ and where ${V}_{{\bf q}}$ is the Fourier transform of $V({\bf x})$. 


\clearpage
    
