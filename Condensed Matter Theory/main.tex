\documentclass{homework}
\usepackage[subpreambles=true]{standalone}
\usepackage{import}
\usepackage{float}
\usepackage{xr}

\externaldocument{1. Quantum Phase Transitions.tex}
\externaldocument{2. Spin Chain Models.tex}
\externaldocument{3. Ising and Heisenberg Spin Models.tex}
\externaldocument{4. Spin Wave Theory.tex}
\externaldocument{5. Additional Topics.tex}
\externaldocument{6. Equilibrium Quantum Field Theory.tex}
\externaldocument{7. Non-Equilibrium Quantum Field Theory.tex}
\externaldocument{8. QFTaCMT.tex}
\externaldocument{9. Path Integrals in CMT.tex}
\externaldocument{10. Bosonic Systems.tex}
\externaldocument{11. Fermionic Systems.tex}

\author{Tomás Pérez}
\class{Condensed Matter Theory - Lecture Notes}
\date{\today}
\title{Theory \& Notes}
\graphicspath{{./media/}}

\begin{document} \maketitle

\tableofcontents

%\iffalse
\section{{\textbf{Quantum Phase Transitions}}}
\import{sections/}{1. Quantum Phase Transitions}
\clearpage

\section{{\textbf{Spin Chain models}}}
\import{sections/}{2. Spin Chain Models}
\clearpage

\section{{\textbf{Ising and Heisenberg Models}}}
\import{sections/}{3. Ising and Heisenberg Spin Models}
\clearpage

\section{{\textbf{Spin Wave Theory}}}
\import{sections/}{4. Spin Wave Theory}
\clearpage

\section{{\textbf{Additional Topics}}}
\import{sections/}{5. Additional Topics}
\clearpage
%\fi 

\section{Equilibrium vs. Non-Equilibrium Quantum Field Theories} 

\import{sections/}{6. Equilibrium Quantum Field Theory}

%\iffalse

\import{sections/}{7. Non-Equilibrium Quantum Field Theory}

\section{{\textbf{Equilibrium Quantum Field-Theoretic approach to Condensed Matter Physics}}}
\import{sections/}{8. QFTaCMT}

\section{{\textbf{Path Integral Formalism and its applications in Condensed Matter Theory}}}
\import{sections/}{9. Path Integrals in CMT}

\section{\textbf{Applications to Bosonic and Fermionic Quantum Many-Body Systems}}

\subsection{Partition Function of Ideal Quantum Systems}

The partition function of a general quantum many-body system is given in terms of a gaussian/coherent state integral of the action, as stated in  \cref{PathIntegral_Partition_function_gaussian_states}. In particular, the action reads 

\begin{equation}
    \begin{split}
        &S_0 = \sum_{\alpha, n} \xoverline{\eta}_{\alpha, n} [-i \nu_n + (\epsilon_{\alpha} - \mu)] \eta_{\alpha, n}, \textnormal{ where $\nu_n$ is a Matsubara frequency}. \\
        &\Rightarrow \mathcal{Z}_0 = \lim_{N \rightarrow \infty} \int d[\xoverline{\bm{{\psi}}}, \bm{\psi}] \int_{\substack{
            \bar{\bm{\psi}}_N = \zeta \bar{\bm{\psi}}_0 
            \\
            {\bm{\psi}}_N = \zeta {\bm{\psi}}_0 
        }} \bigg(
                \prod_{n=1}^{N=1} \prod_{\alpha}        d\bar{\psi}_{\alpha, n} d{\psi}_{\alpha, n}
            \bigg) 
            \exp \bigg[-\delta \beta \sum_{n=1}^{N-1} \sum_{\alpha} 
                \frac{\bar{\psi}_{\alpha, n} {\psi}_{\alpha, n}}{\delta \beta}
            \bigg] \\
            & \qquad \qquad \qquad \times \exp \bigg[
                \sum_{n=1}^{N-1}
                \bigg(
                    \frac{\bar{\psi}_{\alpha, n} {\psi}_{\alpha, n}}{\delta \beta} - {\bf H}(\bar{\bm{\psi}}_{n}, \bm{\psi}_{n-1}) + \mu {\bf N} (\bm{\bar{\psi}}_{n}, \bm{\psi}_{n-1})
                \bigg)
            \bigg] \\
        & \textnormal{In this discretized form, the action $S_0$ - the term inside the second exponential - reads as } \\
        &S_0 =  \sum_{\alpha, m} \bigg[             
                        \xoverline{\eta}_{\alpha, m}            (\eta_{\alpha, m} - \eta_{\alpha, m-1} ) - \delta \beta (\varepsilon_{\alpha} - \mu) 
                        \xoverline{\eta}_{\alpha, m} \eta_{\alpha, m-1}
        \bigg] \\
    \end{split}
\end{equation}

Considering a Matsubara-Fourier series expansion of the form

\[
    \eta_{\alpha, m} = \frac{1}{\sqrt{N}} \sum_{n \in \N} a_{\alpha,n} e^{-i\omega_m \tau_m},
\]

in which the $a_{\alpha, n}$ are c-numbers in the bosonic case and Grasmann variables in the fermionic case. 
Hence, the action reads 

\begin{equation}
    \begin{split}
    \mathcal{Z}_0 &= \lim_{N \rightarrow \infty} \int d[\xoverline{\bm{{\psi}}}, \bm{\psi}] 
                \prod_{n=1}^{N=1} \prod_{\alpha}        
                d({\bf a}^{*}_n, {\bf a}_n)
                \exp \bigg[
                    - \sum_{\alpha} \sum_{n = 0}^{N-1}
                        a_{\alpha, n}^{*} a_{\alpha, n} 
                        (1 + (\delta \beta (\varepsilon_{\alpha} - \mu)- 1)) e^{i \nu_n \delta \beta}
                \bigg] \\
    &= \lim_{N \rightarrow \infty} \prod_{\alpha} \prod_{n=0}^{N-1} \bigg[(1 + (\delta \beta (\varepsilon_{\alpha} - \mu)- 1)) e^{i \nu_n \delta \beta}\bigg]^{-\zeta}, \quad \textnormal{Writing $1 - \delta \beta (\varepsilon_{\alpha} - \mu) \simeq e^{- \phi_{\alpha}}$}, 
    \end{split}
\end{equation}

\begin{equation*}
    \begin{split}
    = \lim_{N \rightarrow \infty} \prod_{\alpha} \prod_{n=0}^{N-1} [(1 - \zeta e^{i \nu_n \delta \beta - \phi} )] = \lim_{N \rightarrow \infty}
        \prod_{\alpha} 
        [(1 - \zeta e^{-N \phi_{\alpha}})]^{-\zeta} \\
    = \prod_{\alpha} \bigg(1- \zeta e^{-\beta (\varepsilon_{\alpha} - \mu)} \bigg)^{-\zeta},
    \end{split}
\end{equation*}

which is the partition function for a non-interacting bosonic/fermionic system. 

\import{sections/}{10. Bosonic Systems.tex}

\section{End}
\clearpage

\import{sections/}{11. Fermionic Systems.tex}
%\fi

\clearpage
\bibliography{ref.bib}
\bibliographystyle{apsrev4-1}

\end{document}
\clearpage

%\section{Linear Response Theory}

Linear response theory is an extremely widely used concept in physics, stating that the response to a weak external perturbation is proportional to the perturbation, and therefore the quantity of interest is the proportionality constant. The physical question to ask is thus: supposing some perturbation $H'$, what is the measured consequence for an observable quantity ${\bf A}$. In other words, what is $\langle {\bf A} \rangle$ to linear order in $H'$? 

Among the numerous physical application of the linear response formalism, one can mention charge and spin susceptibilities of eg. electron systems due to external electric or magnetic fields. Responses to external mechanical forces or vibrations can also be calculated using the same formalism. \\

%\paragraph{\textbf{The general Kubo formula}}

Consider a quantum system described by a time independent Hamiltonian ${\bf H}_0$ in thermodynamic equilibrium. This means that an expectation value of a physical quantity, described by the operator ${\bf A}$, which can be evaluated as 

\begin{equation}
    \begin{split}
        \langle {\bf A} \rangle = \frac{1}{\mathcal{Z}_0} \textnormal{Tr }_{\mathds{H}} \rho_0 {\bf A}  = \frac{1}{\mathcal{Z}_0} \sum_{n \in \Lambda} \bra{n} {{\bf A}} \ket{n} e^{-\beta E_n} \\
        \rho_0 = e^{-\beta {\bf H}_0} = \sum_{n \in \Lambda} \ket{n} \bra{n} e^{-\beta E_n},
    \end{split} \begin{array}{c}
         \textnormal{ where $\{\ket{n}\}_{n \in \Lambda}$ is a complete} \\
         \\
         \textnormal{set of eigenstates}.
    \end{array}
\end{equation}

Suppose now that at some time, $t = t_0$, an external perturbation is applied to the system, driving it out of equilibrium. The perturbation is described by an additional time dependent term in the Hamiltonian 

\begin{equation}
    \Hamiltonian(t) = \Hamiltonian_0 + \Hamiltonian'(t) \theta(t-t_0)
\end{equation}

Now, the interest lies in finding the expectation value of the ${\bf A}$ operator at times $t$ greater that $t_0$-. In order to do so, the time evolution of the density matrix must be found, or equivalently the time evolution of the eigenstates of the unperturbed Hamiltonian. Once $\ket{n(t)}$ is found, the time-dependent expectation value can be found as 

\begin{align}
        \langle {\bf A}(t) \rangle = \frac{1}{\mathcal{Z}_0} \textnormal{Tr }_{\mathds{H}} \rho(t) {\bf A}  = \frac{1}{\mathcal{Z}_0} \sum_{n \in \Lambda} \bra{n(t)} {{\bf A}} \ket{n(t)} e^{-\beta E_n} \\
        \rho_0 = e^{-\beta {\bf H}_0} = \sum_{n \in \Lambda} \ket{n(t)} \bra{n(t)} e^{-\beta E_n},
\end{align}

The physical idea behind this expression is as follows. The initial states of the system are distributed according to the usual Boltzmann distribution $\frac{e^{-\beta E_{0n}}}{\mathcal{Z_0}}$. At later times, the system is described by the same distribution of states but the states are now time-dependent and they have evolved according to the new Hamiltonian. The time dependence of the states $\ket{n(t)}$ is governed by the Schr\"odinger equation. Since $\Hamiltonian'$ is regarded to be a small perturbation, the interaction picture representation is suitable for this setting. In this representation, the time dependence is given by 

\begin{equation}
    \ket{n(t)} = e^{-i\Hamiltonian_0 t} \ket{n(t)}_I = e^{-i\Hamiltonian_0 t} \mathcal{U}(t, t_0) \ket{\hat{n}(t_0)},
\end{equation}

where by definition $\ket{\hat{n}(t_0)} = e^{i \Hamiltonian_0 t_0} \ket{n(t_0)} = \ket{n}$.

Up to linear order in $\Hamiltonian'$, the time evolution operator $\mathcal{U}(t, t_0)$ can be written 

$$
    \mathcal{U}(t, t_0) = \mathds{1} - i \int_{t_0}^t dt' \Hamiltonian'(t') + \mathcal{O}(\Hamiltonian^{'2})
$$

Then, using this in the time-dependent expectation value yields 

\begin{equation} \begin{split}
    \langle {\bf A}(t) \rangle &= \frac{1}{\mathcal{Z}_0} \sum_{n \in \Lambda} \bra{n(t)} {{\bf A}} \ket{n(t)} e^{-\beta E_n} \\
    &= \frac{1}{\mathcal{Z}_0} \sum_{n \in \Lambda} \bra{n(t_0)} \bigg(\mathds{1} - i \int_{t_0}^t dt' \Hamiltonian'(t')\bigg) e^{-i \Hamiltonian_0 t_0} {{\bf A}} e^{i \Hamiltonian_0 t_0} \bigg(\mathds{1} - i \int_{t_0}^t dt' \Hamiltonian'(t')\bigg) \ket{n(t_0)}  e^{-\beta E_n}+ \mathcal{O}(\Hamiltonian^{'2}) \\
    &= \frac{1}{\mathcal{Z}_0} \sum_{n \in \Lambda} \bra{n(t_0)} {{\bf A}} \ket{n(t_0)} - \frac{i}{\mathcal{Z}_0} \int_{t_0}^{t} dt' \sum_{n \in \Lambda} e^{-\beta E_n} \bigg( e^{-i \Hamiltonian_0 t_0} {{\bf A}} e^{i \Hamiltonian_0 t_0} \Hamiltonian'(t') - \Hamiltonian'(t')e^{-i \Hamiltonian_0 t_0} {{\bf A}} e^{i \Hamiltonian_0 t_0} \ket{n(t_0)} \bigg) \\
    &= \langle {{\bf A}} \rangle_{0} + \int_{t_0}^{t} \frac{dt'}{i} \langle[{{\bf A}}(t), \Hamiltonian'(t')]\rangle_{0},
\end{split}
\end{equation}

where the brackets $\langle \rangle_{0}$ means an equilibrium average with respect to the Hamiltonian $\Hamiltonian$. This is in fact a remarkable and very useful result since the inherently non-equilibrium quantity $\langle {{\bf A}}(t) \rangle$ has been expressed as a correlation function of the system in equilibrium. The physical reason for this is that the interaction between excitations created in the non-equilibrium state is an effect to second order in the weak perturbation, hence not included in the linear response. \\

The correlation function is the retarded correlation function, which can be rewritten as 
the difference between $\langle {{\bf A}}(t) \rangle \textnormal{ and } \langle {{\bf A}}\rangle_0 $ ie. 

\begin{align} 
        & \alignedbox{\delta \langle {{\bf A}}(t) \rangle }{
        = \int_{t_0}^{\infty} dt' \mathcal{C}_{AH'}^{R}(t,t') e^{-\eta(t-t')} \textnormal{ where } \mathcal{C}_{AH'}^{R}(t,t') = -i \theta(t-t') \langle[{{\bf A}}(t), \Hamiltonian'(t')]\rangle_{0}},
\end{align}

which is the Kubo formula. This expresses the linear response to a perturbation $\Hamiltonian'$. Note that the factor $e^{-\eta(t-t')}$, with an infinitesimal positive parameter $\eta$, has been included to force the response at time $t$ due to the influence of $\Hamiltonian'$ at time $t'$ to decay when $t >> t'$. At the end of the calculation, the limit $\eta \rightarrow 0^+$,. This is so since the retarded effect of a perturbation must decrease in time\footnote{The other, the advanced correlation function is non-physical and must be ditched. The other one, the retarded correlation function, decreases exponentially with time, the exponential factor thus picks out the physically relevant solution by introducing an artificial relaxation mechanism.}. \\

%\paragraph{\textbf{Kubo fromula in the frequency domain}}

It is often convenient to express the response to an external disturbance in the frequency domain via Fourier transformations\footnote{Consider the $L^1$-space, ie. the space of all integrable functions on the real line. Then the Fourier transform and the Fourier-anti transform can be defined as 
\begin{align}
    \mathcal{F}[f(t)](\omega) = f(\omega) = \int_{\R} dt e^{i\omega t} f(t) \blanky \textnormal{ and } \blanky \mathcal{F}^{-1}[f(\omega)](t) = f(t) = \int_{\R} \frac{d\omega}{2\pi} e^{-i\omega t} f(\omega)
\end{align}

}. Therefore, consider the perturbation Hamiltonian $\Hamiltonian'$, which can be rewritten in terms of its Fourier components 

\begin{equation}
    \Hamiltonian'(t) = \int_{\Omega \subset \R} \frac{d\omega}{2\pi} e^{-i\omega t} \Hamiltonian'_\omega,
\end{equation}

such that the retarded correlation function becomes 

\begin{equation}
    \mathcal{C}_{AH'}^{R}(t,t') = \int_{\mathds{R}} \frac{d\omega}{2\pi} e^{-i\omega t'} \mathcal{C}_{AH'_\omega}^{R}(t-t') \begin{array}{c}
         \textnormal{since $\langle[{{\bf A}}(t), \Hamiltonian'(t')_{\omega'}]\rangle_{0}$ only depends } \\
         \textnormal{ on the difference between $t$ and $t'$. }
    \end{array}
\end{equation}

Therefore, inserting this result into the Kubo formula yields

\begin{equation}
    \begin{split}
        \delta \langle {{\bf A}}(t) \rangle
        &= \int_{t_0}^{\infty} dt' \mathcal{C}_{AH'_\omega}^{R}(t,t') e^{-\eta(t-t')} \\
        &= \int_{\R} dt' \int_{\R} \frac{d\omega}{2\pi} e^{-i\omega t} e^{-i(\omega + i\eta) (t'-t)} \mathcal{C}_{AH'_\omega}^{R}(t-t') \\
        &= \int_{\R} \frac{d\omega}{2\pi} e^{-¿i\omega t} \bigg(\int_{\R} d(t'-t) e^{-i(\omega + i\eta)(t'-t)} \mathcal{C}_{AH'_\omega}^{R}(t-t')\bigg) \\
        &= \int_{\R} \frac{d\omega}{2\pi} e^{-i\omega t} \mathcal{C}_{AH'_\omega}^{R}(\omega; \eta) 
    \end{split}
\end{equation}

which can be inverted to yield the final result in the frequency domain

\begin{equation}
    \begin{split}
        \int_{\R} {dt} e^{i\nu t}
        \delta \langle {{\bf A}}(t) \rangle &= \int_{\R} {dt} e^{i\nu t} \int_{\R} \frac{d\omega}{2\pi} e^{-i\omega t} \mathcal{C}_{AH'_\omega}^{R}(\omega; \eta) \\
        \langle {{\bf A}}_\nu \rangle &= \int_{\R} \frac{d\omega}{2\pi} \int_{\R} dt e^{i(\nu - \omega)t} \mathcal{C}_{AH'_\omega}^{R}(\omega; \eta) \\
        &= \int_{\R} {d\omega} \delta(\nu - \omega) \mathcal{C}_{AH'_\omega}^{R}(\omega; \eta) \\
        &= \mathcal{C}_{AH'_\nu}^{R}(\nu; \eta) 
    \end{split}
\end{equation}

\begin{align}
       & \alignedbox{\Rightarrow \delta \langle {{\bf A}}_\omega \rangle = \mathcal{C}_{AH'_\omega}^{R}(\omega) \textnormal{ with } \mathcal{C}_{AH'_\omega}^{R}(\omega)}{ = \int_{\R} dt e^{i\omega t} e^{-\eta t} \mathcal{C}_{AH'_\omega}^{R}(t)},
\end{align}

{where the infinitesimal $\eta$ parameter is incorporated } { in order to ensure the correct physical result, ie.} {the retarded response function decays at $t>>1$.} \\

%\paragraph{\textbf{Kubo formula for conductivity}}

Consider a system of charged particles, eg. electrons, which is subjected t an external electromagnetic field. The electromagnetic field induces a current and the conductivity is the linear response coefficient. In the general case, the conductivity is a non-local quantity in both time and space, such that the electric current ${{\bf J}_e}$ at some point ${\bf x}$ at time $t$ depends on the electric field at points ${\bf y}$ at times $t'$, eg.\footnote{Note that this section's mathematical treatment is not Lorentz-covariant. This is, the spacetime is treated as $\R^4$ with the metric being $g_{\mu \nu} = \delta_{\mu \nu}$} 

\begin{equation}
    J^{\alpha}_{e}(x^\mu) = \int_{\Omega \subset \R^4 } dx^{\mu} \sum_{\alpha \beta} \sigma_{\beta}(x^\mu, y^{\mu}) E^{\beta} (y^{\mu}), \begin{array}{c} 
         \textnormal{ where $\sigma_{\alpha \beta}(x^\mu, y^{\mu})$ is the conductivity tensor, which  } \\ 
         \textnormal{ describes the current response in the } \\
         \textnormal{ ${\bf e}_\alpha$-direction to an applied }
         \textnormal{ electric field in the ${\bf e}_\beta$-direction.}
    \end{array}
\end{equation}

The electric field ${\bf E}$ is given by the electric potential $\phi_{ext}$ and the vector potential ${\bf A}_{\textnormal{ext}}$ as $
{\bf E}(x^\mu) = -\partial_{\mu} A^{\mu}$. For electrons, the current density can be written as ${\bf J}_e = -e \langle {\bf J} \rangle$. The perturbing term in the Hamiltonian due to the external electromagnetic field is given by the coupling of the electrons to both the scalar potential and the vector potential. Then, upto linear order in the external potential, 

$$
\Hamiltonian_{\textnormal{ext}} = -e \int_{\R^3} d{\bf x} J_{\mu}({\bf x}) A^{\mu}_{\textnormal{ext}}(x^{\mu}).
$$

Let, ${\bf A}_0$ denote the vector potential in the equilibrium ie. prior to the onset of the perturbation ${\bf A}_0(x^{\mu}$ and let ${\bf A}_0(x^{\mu})$ denote the total vector potential. Then, 

$$
   {\bf A}(x^{\mu}) = {\bf A}_{0}(x^{\mu}) +  {\bf A}_{\textnormal{ext}}(x^{\mu}).
$$

The current operator can be decomposed in two components, the diamagnetic and the paramagnetic terms, as follows 

\begin{equation}
    {\bf J} = {\bf J}^{\nabla}({\bf x}) + \frac{e}{m} {\bf A}({\bf x}) \rho({\bf x}).  
\end{equation}

For simplicity and using gauge invariance, the external electric potential can be set to zero. The conductivity is most easily expressed in the frequency domain via a Fourier transformation of the perturbation. Since 

\begin{equation} \begin{array}{cc}
     \partial_t \overset{\mathcal{F}}{\rightarrow} -i\omega \blanky & \textnormal{ then } \blanky {\bf A}_{\textnormal{ext}}({\bf x}, \omega) \overset{\mathcal{F}}{\rightarrow}  \frac{1}{i\omega} {\bf E}_{\textnormal{ext}}({\bf x}, \omega) \\
\end{array}
     \Rightarrow \Hamiltonian_{\textnormal{ext}, \omega} = \frac{e}{i\omega} \int_{\Omega \subset \R^3} d{\bf x} \blanky {\bf J}({\bf x}) \cdot {\bf E}_{\textnormal{ext}}({\bf x}, \omega).
\end{equation}

In order to exploit the frequency domain formulation of linear response theory, it is desirable to find the corresponding formula for the conductivity tensor in frequency-space. The conductivity tensor is a property of the equilibrium system and can thus onyly depend on time differences $ \sigma_{\alpha \beta}(x^\mu, y^{\mu}) =  \sigma_{\alpha \beta}({\bf x},{\bf y}, t-t')$. The frequency transform of the conductivity yields

\begin{equation}
    J^{\alpha}_{e}({\bf x}, \omega) = \int_{\Omega \subset \R^3 } d{\bf y} \blanky \sum_{ \beta} \sigma_{\alpha \beta}({\bf x},{\bf y}, \omega) E^{\beta}({\bf y}, \omega).
\end{equation}

Now, given that he external perturbation, written in frequency space, is already linear in the external potential ${\bf E}_{\textnormal{ext}}$, and given that the interest lies only on terms proportional to said perturbation, the conductivity can be rewritten as 

\clearpage

%\section{Microscopic Theory of Conventional Superconductivity}

From \textbf{Advanced Quantum Condensed Matter Physics, One-Body, Many-Body and Topological perspectives. Michael El-Batanouny} \\

%\paragraph{\textbf{}}

\end{document}

