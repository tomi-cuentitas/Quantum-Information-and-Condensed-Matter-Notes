\documentclass{homework}
\author{Tomás Pérez}
\class{Density Functional Theory - Lecture Notes}
\date{\today}
\title{Theory \& Notes}

\graphicspath{{./media/}}

\begin{document} \maketitle

\section{Quantum Many-Body Problem}

In this section, we're interested in the non-relativistic quantum many body problem, where the system of study is made up of electrons and nuclei. The theory's Hamiltonian is given by

\begin{align}
    \hat{\bf H} = -\frac{\hbar^2}{2m_e} \sum_i \nabla^2_i - \sum_{i, I} \frac{Z_I e^2}{|{\bf x}_i - {\bf R}_I|} + \frac{1}{2} \sum_{i \neq j} \frac{e^2}{|{\bf x}_i-{\bf x}_j|} - \sum_{I} \frac{\hbar^2}{2M_I} \nabla^2_I + \frac{1}{2} \sum_{I \neq J} \frac{Z_I Z_J e^2}{|{\bf R}_I - {\bf R}_J|},
\label{many-body Hamiltonian}
\end{align}

where 

\begin{itemize}
    \item the first corresponds to the sum of the kinetic energy of each one of the electrons,
    \item the second term correspond to the Coulomb interaction created by the nuclei, where the position of $i$-th electron is ${\bf x}_i$ and the nuclei's position is ${\bf R}_I$,
    \item the third term is the Coulomb interaction between the electrons, where we exclude any auto-interaction of an electron with itself. Note that this term forces to consider more than one electron at a time. 
    \item The fourth term is the kinetic energy of the nuclei,
    \item and where the last term accounts for the nucleus-nucleus interaction. \\
\end{itemize}

Solving the general $n$-body problem consists then in finding the eigenfunctions and eigenvalues of the coupled, second order, differential equation given by the action of \eqref{many-body Hamiltonian} on the $n$-body wave function. However, by inspecting the terms of the Hamiltonian we can simplify the problem. Note that the mass of the nuclei is much larger than the mass of the electrons, therefore we can neglect their kinetic energy given by \eqref{many-body Hamiltonian}'s fourth term. Likewise, if we consider the position of the nuclei as parameters, the fifth term adds just a constant. Taking the previous points into consideration, yields a more simplified Hamiltonian:

\begin{equation}
    \hat{\bf H} = -\frac{\hbar^2}{2m_e} \sum_i \nabla^2_i - \sum_{i, I} \frac{Z_I e^2}{|{\bf x}_i - {\bf R}_I|} + \frac{1}{2} \sum_{i \neq j} \frac{e^2}{|{\bf x}_i-{\bf x}_j|} + \sum_{i} {\bf V}_i ({\bf x}_i).
\label{simplified many-body Hamiltonian}
\end{equation}

Note that in said equation, in the second term, we considered only the Coulomb interaction between the electrons and the nuclei. We might as well add any other external potentials acting on the electrons, including the ones stemming from the nuclei, which are accounted for in the fourth term. \\

This is still a highly-non trivial problem to solve. Indeed, solving said problem consists of solving the eigenvalue equation:

\begin{equation}
\hat{\bf H}({\bf x}_1, \cdots, {\bf x}_N) \Psi_\lambda(\tilde{x}_1, \cdots, \tilde{x}_N) = E_\lambda \Psi_\lambda(\tilde{x}_1, \cdots, \tilde{x}_N),
\label{ev equation}
\end{equation}

where 

\begin{itemize}
    \item $\tilde{x}_i = ({\bf x}_i, \sigma_i)$, is an atlas on a Riemannian manifold $V = \R^{3N} \otimes (\mathds{C}^2)^{\otimes N}$, accounting for both the $i$-th electron's position and spin,
    \item and where $\Psi_\lambda$, describing the wavefunction of the $N$-body system, is a function in a Hilbert space given by $\mathds{H} \approx L^2(\R^{3N}) \otimes \bigotimes_{j=1}^{N} \mathds{C}^2 \approx L^2(\R^{3N}) \otimes (\mathds{C}^{2})^{\otimes N}$\footnote{Remember that the Hilbert space for a particle in $\R^3$ with spin is the tensor product of $L^2(\R^3)$ with a finite-dimensional vector space $V$, where $V$ carries and irreducible action of the group rotation $SO(3)$. In this context, the proper notion of "action" is the projective representation of $SO(3)$, meaning a family of operators satisfying the SO(3)-Lie algebra, $\mathfrak{so}(3)$. In our context, the full Hilbert space of a system of $N$ spin-$s$ particles is isomorphic to the tensor product of $N$-Hilbert spaces $\mathds{C}^2$. This may be written in short as $(\mathds{C}^{2s+1})^{\otimes N}$}. Even if we ignore the spatial degrees of freedom, the spin-Hilbert space's dimension scales up as $2^N$. Thus, this problem becomes an NP-hard problem. This is even worse given that we cannot decouple the coordinates of the $i$-th electron from the $j$-th electron. \\ 
\end{itemize}

One way to simplify \eqref{ev equation} is by disregarding the Coulomb interaction and the Pauli principle, considering a system of non-interacting particles. This turns said eigenvalue equation into, a priori, $N$-single electron uncoupled problems, ie.

$$
\Psi_\lambda(\tilde{x}_1, \cdots, \tilde{x}_N) = \prod_{i=1}^{N} \psi_{n_i}(\tilde{x}_i), \textnormal{with } \psi_{n_i}(\tilde{x}) \in L^2(\R^{3}) \otimes \mathds{C}^2 \textnormal{ and } \lambda = \{n_i\}_{i=1}^{N}.
$$

This problem, while tractable, is unphysical since it implies that all electrons move independently in the external potential, allowing for multiple electrons to be in the fundamental state (or any other state for that matter), which goes against the Pauli principle. One way in which we can take into account the Pauli Principle is by taking the Slater determinant of the $N$ single-electron wave functions. Even so, the electrons would still be independent. \\

\section{Observables}

The general purpose for solving the eigenvalue equation \eqref{ev equation} is to then use said wave-function to calculate observables. Consider the grand canonical ensemble and a given operator $\hat{\bf O} : \mathds{H} \rightarrow \mathds{H}$, then 

\begin{equation}
\langle \hat{\bf O} \rangle = \frac{1}{\mathcal{Z}} \sum_{\alpha \in \Lambda} e^{-\beta(E_\alpha - \mu N_\alpha)} \bra{\Psi_\alpha}\hat{\bf O}\ket{\Psi_\alpha},
\label{grand canonical}
\end{equation}

where 

\begin{itemize}
    \item $\Lambda$ indexes the different states,
    \item where $\beta = \frac{1}{kT}$ and $\mu$ is the chemical potential,
    \item $E_\alpha$ and $N_\alpha$ are the energy and particle number in state $\ket{\Psi_\alpha}$ respectively,
    \item and where $\mathcal{Z} =  \sum_{\alpha \in \Lambda} e^{-\beta(E_{\alpha} - \mu N_{\alpha})}$ is the partition function. 
\end{itemize}

In other words, \eqref{grand canonical}, the operator $\hat{\bf O}$ stands for the measurement and is integrated with all the many-body states $\ket{\Psi_\alpha} \in \mathds{H} \approx L^2(\R^{3N}) \otimes (\mathds{C}^{2})^{\otimes N}$ which describe the system. \\

At $N$ fixed and a zero temperature, \eqref{grand canonical} means:

\begin{equation}
     \langle \hat{\bf O} \rangle = \int_{\mathds{R}^{3N}} \prod_{i=1}^N d{\bf x}_i \prod_{j=1}^{N} \sum_{\sigma_j \in \mathds{C}^2} \Psi^{*}(\tilde{x}_1, \cdots, \tilde{x}_N) \sum_{a,b,\cdots} O(x_a, x_b, \cdots) \Psi(\tilde{x}_1, \cdots, \tilde{x}_N),
\label{ensemble avg}
\end{equation}

which can be thought as a functional of the many-body wave function, $\langle \hat{\bf O} \rangle = \langle \hat{\bf O} \rangle[\Psi]: {V}^{\mathds{H}} \rightarrow \R$, with $V = \R^{3N} \times \mathds {C}^{2N}$. Now, note that many details of the many-body wavefunction are integrated out. \\

One way to resolve the many-body problem consists in using an approximate wave-function. For example, Hartree-Fock, configuration interaction and Quantum Monte-Carlo are based in this approach. Another, more radical approach, consists in asking ourselves if we really need to work with all possible many-body wave-functions. \\

Note that \eqref{ensemble avg} depends, at most, with the form of the Hamiltonian, which in our case is given by 
 
\begin{equation}
    \hat{\bf H} = -\frac{\hbar^2}{2m_e} \sum_i \nabla^2_i + \sum_{i} {\bf V}_{\textnormal{ext}} ({\bf x}_i) + \frac{1}{2} \sum_{i \neq j} \frac{e^2}{|{\bf x}_i-{\bf x}_j|}.
\label{hamiltonian 3}
\end{equation}

We consider that the kinetic energy operator and the Coulomb interaction contain enough information to completely determine the value of an observable. Therefore, we can consider the expectation value to be a functional on the external potential, instead of a functional of the many-body wave-function:

$$
\langle \hat{\bf O} \rangle = \langle \hat{\bf O}  \rangle[\Psi] \Rightarrow
\langle \hat{\bf O}  \rangle[V_{\textnormal{ext}}]
: {V}^{\mathds{H}} \rightarrow \R.
$$

Thinking this in terms of functionals, eliminate the explicit dependence on the coordinates $(\tilde{x}_1, \cdots, \tilde{x}_n)$. In order to understand this new functional we note, that in the grand canonical ensemble, the sum of states is actually a trace over the Hilbert space's states, and since the trace is invariant, we can re-write \eqref{ensemble avg} as 

\begin{equation}
    \langle \hat{\bf O} \rangle = \frac{1}{\mathcal{Z}} \Tr_{\mathds{H}} \bigg(e^{-\beta(\hat{\bf H}-\mu\hat{\bf N})} \hat{\bf O} \bigg),
\label{trace gce}
\end{equation}

with the Hamiltonian given by \eqref{hamiltonian 3}. Therefore \eqref{trace gce} depends only on the external potential. Thus, we have expressed a rather complicated functional in terms of a simple function, which we can then try to approximate.

\section{Subsets of interesting observables}

Consider an electronic structure, what are the typical observables of interest? \\

A first example could be the equilibrium distance between two atoms in a molecule, which is the result of an energy-minimization problem. The energy minimization can also give the full geometry of a molecule. 
For example, 

\begin{itemize}
    \item consider a carbon dioxide molecule, the energy minimization problem predicts a linear structure with a flat angle between the two oxygens. Likewise, for the water molecule, the energy minimization problem predicts a specific angle between the hydrogens in said molecule. The same can be said about the ammonia, which is not a planar molecule. 
    \item The crystal structure can also be predicted in this way, by evaluating the forces acting on each atom of the system.
    \item In particular, this method can also compute the lattice vibrations that give rise to phonons or the bulk modulus, the ability of a certain material to resist to a certain external pressure. \\
\end{itemize}

These are very specific examples in which the core of problem is to evaluate the ground-state energy. In DFT, even the excited states of a system are observables as well as some response functions, such as optical absorption or electron energy loss. \\

\section{Observables in terms of Compact Quantities}

This section's aim is to find a simple quantity, in terms of which we can express the observables. Let $(\mathcal{B}(\mathds{H}), ||\cdot||)$ denote the space of all linear operators acting on the Hilbert space, noting that the sub-space of all linear bounded operators is a Banach space. Let $\mathcal{B}_1(\mathds{H})=\{\hat{\bf O}| \hat{\bf O}:\mathds{H}_k \rightarrow \mathds{H}_k \}$ ie. the space of all one-body operators, then we define the space of $k$-body operators as 

$$
\mathcal{B}_k(\mathds{H}) = \{\otimes_{i=1}^{k} {\bf O}_i | {\bf O}_i \in \mathcal{B}_1(\mathds{H}) \},
$$

where, obviously,

$$
\mathcal{B}(\mathds{H}) = \bigsqcup_{i=1}^{N} \mathcal{B}_i(\mathds{H}).
$$

Now, consider a one-body operator $\hat{\bf Q}_i \in \mathcal{B}_{1}(\mathds{H})$, which acts on the $i$-th electron's Hilbert space $\mathds{H}_i$, which we desire to calculate its expected value following equation \eqref{ensemble avg}. Therefore, we can construct a new operator 
$$
\hat{\bf Q} = \bigoplus_{i=1}^{N} \hat{\bf Q}_i = \sum_{i=1}^{N} \tilde{\bf Q}_i
$$

where now both $\hat{\bf Q}$ and $\tilde{\bf Q}_i$ act on the full Hilbert space. Note that $\tilde{\bf Q}_i$ acts trivially on all but one of the individual Hilbert spaces:

\begin{align}
     \langle \hat{\bf Q} \rangle &= \int_{\mathds{R}^{3N}} \prod_{i=1}^N d{\bf x}_i \prod_{j=1}^{N} \sum_{\sigma_j \in \mathds{C}^2} \Psi^{*}(\tilde{x}_1, \cdots, \tilde{x}_N)
     \sum_{i}
     \tilde{\bf Q}_i(\tilde{x_i})
     \Psi(\tilde{x}_1, \cdots, \tilde{x}_N) \\
     &= N \int_{\mathds{R}^{3N}} \prod_{i=1}^N d{\bf x}_i \sum_{\sigma_1 \in \mathds{C}^2} \Psi^{*}(\tilde{x}_1, \cdots, \tilde{x}_N)
     \tilde{\bf Q}_1(\tilde{x}_1)
     \Psi(\tilde{x}_1, \cdots, \tilde{x}_N) \\
     &= \int_{\mathds{R}^{3}} d{\bf x}_1 \sum_{\sigma_1 \in \mathds{C}^2} \tilde{\bf Q}_1(\tilde{x}_1) N \int_{\mathds{R}^{3(N-1)}} \prod_{i=2}^N d{\bf x}_i \prod_{j=2}^{N} \sum_{\sigma_j \in \mathds{C}^2} \Psi^{*}(\tilde{x}_1, \cdots, \tilde{x}_N)
     \Psi(\tilde{x}_1, \cdots, \tilde{x}_N) \\
     &= \int_{\mathds{R}^{3}} d{\bf x}_1 \sum_{\sigma_1 \in \mathds{C}^2} \tilde{\bf Q}_1(\tilde{x}_1) {\bf n}(\tilde{x}_1),
\label{one-body ensemble avg}
\end{align}

where 
 
\begin{equation}
{\bf n}(\tilde{x}_1) = N \int_{\mathds{R}^{3(N-1)}} \prod_{i=2}^N d{\bf x}_i \prod_{j=2}^{N} \sum_{\sigma_j \in \mathds{C}^2} \Psi^{*}(\tilde{x}_1, \cdots, \tilde{x}_N)
     \Psi(\tilde{x}_1, \cdots, \tilde{x}_N),
\label{one-body density}
\end{equation}

is the one-body density. Thus, in order to calculate the expectation value of a one-body operator, we need only the one-body density. Similarly, if we want to compute the expectation value of a general two-body operator, the same steps can be repeated. Consider, for example, the Coulomb interaction. Then it's expectation value is 

\begin{align}
     V_{ee} &= \int_{\mathds{R}^{3N}} \prod_{i=1}^N d{\bf x}_i \prod_{j=1}^{N} \sum_{\sigma_j \in \mathds{C}^2} \Psi^{*}(\tilde{x}_1, \cdots, \tilde{x}_N)
     \sum_{\substack{i,j\\
          j > i}} 
     \frac{e^2}{|{\bf x}_i-{\bf x}_j|}
     \Psi(\tilde{x}_1, \cdots, \tilde{x}_N) \\
     &= \int_{\mathds{R}^{6}} d{\bf x}_1 d{\bf x}_2 \sum_{\sigma_1 \in \mathds{C}^2} \sum_{\sigma_2 \in \mathds{C}^2} \frac{1}{2} \frac{e^2}{|{\bf x}_1-{\bf x}_2|} N(N-1) \int_{\mathds{R}^{3(N-2)}} \prod_{i=3}^N d{\bf x}_i \prod_{j=3}^{N} \sum_{\sigma_j \in \mathds{C}^2} \Psi^{*}(\tilde{x}_1, \cdots, \tilde{x}_N)
     \Psi(\tilde{x}_1, \cdots, \tilde{x}_N) \\
     &=\int_{\mathds{R}^{6}} d{\bf x}_1 d{\bf x}_2 \sum_{\sigma_1 \in \mathds{C}^2} \sum_{\sigma_2 \in \mathds{C}^2} \frac{1}{2} \frac{e^2}{|{\bf x}_1-{\bf x}_2|} {\bf n}^{(2)}(\tilde{x}_1, \tilde{x}_2),
\label{two-body ensemble avg}
\end{align}

where

$$
{\bf n}^{(2)}(\tilde{x}_1, \tilde{x}_2) = N(N-1) \int_{\mathds{R}^{3(N-2)}} \prod_{i=3}^N d{\bf x}_i \prod_{j=3}^{N} \sum_{\sigma_j \in \mathds{C}^2} \Psi^{*}(\tilde{x}_1, \cdots, \tilde{x}_N)
     \Psi(\tilde{x}_1, \cdots, \tilde{x}_N),
$$

is the two-body density.  \\

One interesting observable may be the kinetic  energy, which can readily be written as 

\begin{align}
     T &= -\int_{\mathds{R}^{3N}} \prod_{i=1}^N d{\bf x}_i \prod_{j=1}^{N} \sum_{\sigma_j \in \mathds{C}^2} \Psi^{*}(\tilde{x}_1, \cdots, \tilde{x}_N)
     \sum_{i}\frac{\nabla^{2}_{i}}{2}
     \Psi(\tilde{x}_1, \cdots, \tilde{x}_N) \\
     &= - \int_{\mathds{R}^{3}} d{\bf x}_1 \sum_{\sigma_1 \in \mathds{C}^2} \bigg[\frac{\nabla^{'2}_1}{2} N \int_{\mathds{R}^{3(N-1)}} \prod_{i=2}^N d{\bf x}_i \prod_{j=2}^{N} \sum_{\sigma_j \in \mathds{C}^2} \Psi^{*}(\tilde{x}_1, \cdots, \tilde{x}_N) \Psi(\tilde{x}'_1, \cdots, \tilde{x}_N)\bigg]_{{x'_1 \rightarrow x_1}} \\
     &= \int_{\mathds{R}^{3}} d{\bf x}_1 \sum_{\sigma_1 \in \mathds{C}^2} \frac{\nabla^{'2}_1}{2} {\rho}(\tilde{x}_1, \tilde{x}'_1)_{{x'_1 \rightarrow x_1}},
\end{align}

where, once again, $\rho$ is the one-density matrix 

$$
{\rho}(\tilde{x}_1, \tilde{x}'_1) =  N \int_{\mathds{R}^{3(N-1)}} \prod_{i=2}^N d{\bf x}_i \prod_{j=2}^{N} \sum_{\sigma_j \in \mathds{C}^2} \Psi^{*}(\tilde{x}_1, \cdots, \tilde{x}_N) \Psi(\tilde{x}'_1, \cdots, \tilde{x}_N).
$$

For example, in outlook spectroscopy, other useful ingredients may be the observables associated with transitions to excited states, ie.

\begin{align}
f_{s}^{eh}(\tilde{x}_1, t) &= \int_{\mathds{R}^{3(N-1)}} \prod_{i=2}^N d{\bf x}_i \prod_{j=2}^{N} \sum_{\sigma_j \in \mathds{C}^2} \Psi^{*}(\tilde{x}_1, \cdots, \tilde{x}_N; t) \Psi_{s}(\tilde{x}_1, \cdots, \tilde{x}_N; t) \\
&= e^{-i(E_s-E_0)t} \int_{\mathds{R}^{3(N-1)}} \prod_{i=2}^N d{\bf x}_i \prod_{j=2}^{N} \sum_{\sigma_j \in \mathds{C}^2} \Psi^{*}(\tilde{x}_1, \cdots, \tilde{x}_N) \Psi_{s}(\tilde{x}_1, \cdots, \tilde{x}_N) 
\end{align}

said transition depending on a given number $s$, which indexes the excited state to which the electron has transitioned and where a phase factor has appeared on the second line, an oscillation on the excitation energy. \\

Up to now, in order to calculate the expected value of different $k$-body operators, we need different $N-k$-body kernels, to which integrate them with. Therefore, we first need to calculate said $N-k$-body kernels. However, we insist on finding only one kernel for all different $K-$body operators. \\

%Consider first one-body operators, then the expected values are only functionals of the one-body density given by \eqref{one-body density}.

\section{The Hohenberg-Kohn Theorem}

\subsection{Calculus of Variations}

A functional is a mapping $F[\cdot]: C^{\infty}(\mathds{R})\rightarrow\mathds{R}$. Integrals, for example, can be thought as functionals, values of a function at a certain point $x_0 \in \mathds{R}$ can also be thought as a functional, with a weight function $W(x) = \delta(x-x_0)$. Some other interesting functionals are, for example, 

\begin{itemize}
    \item Local functional: $F[f] = \int_{\mathds{R}} dx \blanky g(f(x))$,
    \item Non-local functionals: 
    $F[f_1, f_2] =\int_{\mathds{R}^2} dx dx' \blanky W(x,x') f_1(x) f_2(x')$, 
    \item the total energy of a many-body system 
    $$
    E = F[\Psi] = \int_{\mathds{R}^{3N}} \prod_{i=1}^{N} d{\bf x}_i \blanky \Psi^{*}({\bf x}_1, \cdots, {\bf x}_N) {\bf H} \Psi({\bf x}_1, \cdots, {\bf x}_N) = \bra{\Psi}{\bf H}\ket{\Psi}.
    $$
    \item The Lagrangian action: $
    S[{\bf q}] = \int dt \blanky \mathcal{L}({\bf q}(t), \dot{\bf q}(t), t).$
\end{itemize}

In the context of the many-body problem, some interesting functionals of the electron density, could be 

\begin{itemize}
    \item The contribution of the external energy given by an external potential
    
    $$
    E_{\textnormal{ext}}[n] = \int_{\mathds{R}^3} d{\bf x}  \blanky n({\bf x}) V_{\textnormal{ext}}({\bf x}),
    $$
    
    which is a local functional.
    \item The classical electrostatic term, also called the Hartree energy, is a non-local functional:
    
    $$
    E_{\textnormal{H}}[n] = \frac{1}{2} \int_{\mathds{R}^6} d{\bf x} d{\bf x}' \blanky \frac{n({\bf x})n({\bf x'})}{|{\bf x}-{\bf x}'|}.
    $$
    
    \item Another interesting functional is the Thomas-Fermi approximation, in which the kinetic energy is a $5/3$-power of the electron density
    
    $$
    T^{TF}[n] = C \int_{\mathds{R}^3} d{\bf x} \blanky n^{\frac{5}{3}}({\bf x}),
    $$
    
    \item or the von Weizs\"acker functional:
    
    $$
    T^{vW}[n] = \frac{1}{8} \int_{\mathds{R}^3} d{\bf x} \blanky \frac{\nabla n({\bf x}) \cdot \nabla n({\bf x})}{n({\bf x})}. 
    $$
\end{itemize}

\blanky \\

\subsection{Functional Derivatives}

In order to construct a functional derivative, we need to consider a generic function $f \in C^{\infty}(\R)$ and a perturbation $\eta(x)$:

\begin{align*}
    F[f(x) + \epsilon \eta(x)] = F[f] + \frac{dF}{d\epsilon}[f+\epsilon \eta]\bigg|_{\epsilon = 0} \epsilon + \frac{1}{2} \frac{d^2 F}{d\epsilon^2}[f+\epsilon \eta]\bigg|_{\epsilon = 0} \epsilon^2 + \mathcal{O}(\epsilon^3).
\end{align*}

Therefore, we can define 

$$
\frac{dF}{d\epsilon}[f+\epsilon \eta]\bigg|_{\epsilon = 0} = \frac{ F[f(x) + \epsilon \eta(x)]  - F[f]}{\epsilon}.
$$

We then define the first and second order functional derivatives as 

\begin{align}
\frac{dF}{d\epsilon}[f+\epsilon \eta]\bigg|_{\epsilon = 0} &= \int_{\R} dx \blanky \frac{\delta F[f]}{\delta f(x)} \eta(x),
& \frac{d^2 F}{d\epsilon^2}[f+\epsilon \eta]\bigg|_{\epsilon = 0} &= \int_{\R^2} dx dx' \blanky \frac{\delta^2 F[f]}{\delta f(x) \delta f(x')} \eta(x) \eta(x').
\end{align}

and so on. Consider now, for example, a local functional of the form

$$
F[f] = \int_{\R_{[a,b]}} dx \blanky g(x, f(x), f'(x)),
$$

which we'd like to calculate its functional derivative. In order to do exactly that, we Taylor-expand the functional in terms of the incremented function 

\begin{align*}
    F[f+\epsilon \eta] = F[f] + \frac{d}{d\epsilon} \int_{\R_{[a,b]}} dx \blanky g(x, f+\epsilon \eta, f'+\epsilon \eta')\bigg|_{\epsilon = 0} \epsilon, \\
    \Rightarrow \frac{d}{d\epsilon} \int_{\R_{[a,b]}} dx \blanky g(x, f+\epsilon \eta, f'+\epsilon \eta')\bigg|_{\epsilon = 0} \epsilon &= \frac{d}{d\epsilon} \int_{\R_{[a,b]}} dx \blanky\bigg[ g(x,f,f') + \epsilon \eta \frac{\partial g}{\partial f}  + \epsilon \eta' \frac{\partial g}{\partial f'} \bigg]\bigg|_{\epsilon = 0} \epsilon \\
    &= \int_{\R_{[a,b]}} dx \blanky\bigg[\eta \frac{\partial g}{\partial f}  + \eta' \frac{\partial g}{\partial f'} \bigg] \epsilon \\
    &= \cancel{\eta \frac{\partial g}{\partial f'}\bigg|_{a}^{b}} \epsilon + \int_{\R_{[a,b]}} dx \blanky\bigg[ \frac{\partial g}{\partial f}  - \frac{d}{dx}\frac{\partial g}{\partial f'} \bigg] \eta (x) \epsilon \\
    &= \int_{\R} dx \blanky \frac{\delta F[f]}{\delta f(x)} \eta(x),\\
\Rightarrow \frac{\delta F[f]}{\delta f(x)} = \frac{\partial g}{\partial f}  - \frac{d}{dx}\frac{\partial g}{\partial f'}, \textnormal{ the Euler-Lagrange equations.} 
 \end{align*}
 
\clearpage

\subsection{Defining the system via an external potential}

Using the previous section's techniques, we'd like to inquire if we can consider a functional of the density as the unique functional of a quantum mechanical observation. Indeed, it's possible to demonstrate that every observable in the ground state is a unique functional of the electron density. This is the main outcome of the Hohenberg-Kohn theorem. \\

We start by saying that every observable is a functional on the many-body wave-function: $\mathcal{O} = \bra{\Psi}\hat{\bf O}\ket{\Psi} = \mathcal{O}[\Psi]$. Now, consider a non-relativistic many-body Hamiltonian in the Born-Oppenheimer, which is given by \eqref{hamiltonian 3}. Note that the most important term in said expression, the operator which distinguishes a many-body problem from another, is the external potential\footnote{For example an atom of nickel contains 28 electrons, the same as the bulk silicon. The structural differences of said many-body systems lie on the different strength and position of the external (nuclei) potentials.}. Since the external potential defines the many-body system, once it's specified, it gives a unique Hamiltonian, which can then be diagonalized to find its eigenvectors and eigenfunctions. \\

Then, if the expectation value of an operator depends on the wave-function, eg. the ground state function, which in turns depends on the Hamiltonian and hence on the external potential, can we say that each observable is uniquely determined by the external potential? In other words, can we say that each observable is a unique functional of the external potential? \\

If that's the case, then there has to be an isomorphism from the space of all external potentials (which can be thought as $C^\infty(\R)$) to the space of all distinct many-body system's wave-functions. This is there has to be a bijection, two different potentials lead to two different wave-functions and vice-versa. Is this the case? At the moment, we cannot say, but if, from a given potential, we obtain a single ground state wave function, thus this mapping is surjection and is then a function. This isn't the case for many-body systems with a degenerate ground state. Then, we can modify our statement as follows: \textit{there is an isomorphism from the space of all external potentials to the space of all wave-functions abiding the same ground state energy}. For all practical purposes, there is a function, an surjective mapping, from the space of all external potentials to the space of all distinct many-body system's wave-functions. \\

It begs the question if this mapping is an injective mapping, should this be the case, there'd be a one-to-one correspondence between the external potential and the wave-function. It would then mean that any observable is a unique functional of the external potential. Another interesting question is if the same argument is valid, not for the wave-functions, but with the one-body densities. The answer to these questions constitute the Hohenberg-Kohn theorem. \\

\subsection{Proof of the Hohenberg-Kohn Theorem}

There are actually two theorems attributed to Hohenberg and Kohn, which relate to any system consisting of electrons moving under the influence of an external potential $v_{\textnormal{ext}}({\bf x})$. In the following discussion, we'll consider only the cases where the mapping, from the space of external potentials to the space of all distinct $N$-electrons many-body groundstate wave-functions, is a function, ie. we exclude the cases of obtaining a degenerate groundstate wave-function. \\

\begin{theo}
The external potential $v_{\textnormal{ext}}({\bf x})$, and hence the total energy, is a unique functional of the electron density $n({\bf x})$.
\end{theo}

\begin{proof}
The energy functional $E[n({\bf x})]$ alluded to in the first Hohenberg-Kohn theorem can be written in terms of the external potential $v_{\textnormal{ext}}({\bf x})$ in the following way 
\begin{equation}
    E_{\textnormal{ext}}[n] = \int_{\mathds{R}^3} d{\bf x}  \blanky n({\bf x}) V_{\textnormal{ext}}({\bf x}) + F[n({\bf x})],
\end{equation}

where $F[n({\bf x})]$ is an unknown, but otherwise universal functional of the electron density $n({\bf x})$ only. Correspondingly, a Hamiltonian for the system can be written in such a way that the electron wave-function, $\Psi$, which minimises the expectation value gives the ground state energy, assuming a non-degenerate groundstate: $E_{\textnormal{ext}}[n] = \bra{\Psi}\hat{\bf H}\ket{\Psi}$. 
The Hamiltonian can then be written as 

$$
\hat{\bf H} = \hat{\bf F} + \hat{V}_{\textnormal{ext}},
$$

where $\hat{\bf F}$ is the electronic Hamiltonian consisting of a kinetic energy operator and an interaction operator. The electron operator $\hat{\bf F}$ is the same for all $N$-electron systems. Therefore, the Hamiltonian ${\bf H}$ is completely defined by the number of electrons $N$ and the external potential $\hat{V}_{\textnormal{ext}}$. The proof of the first theorem lies on considering two different external potentials, say $\hat{V}^{(1)}_{\textnormal{ext}}$ and $\hat{V}^{(2)}_{\textnormal{ext}}$, which we assume to give rise to the same density $n_{0}({\bf x})$. The associated Hamiltonians, labelled $\hat{\bf H}^{(1)}$ and $\hat{\bf H}^{(2)}$, will therefore have different groundstate wave-functions ${\Psi}^{(1)}$ and ${\Psi}^{(2)}$, which yield the same density $n_{0}({\bf x})$. Using the variational principle, we have 

\begin{align}
    E_{0}^{(1)} &< \bra{\Psi_2} \hat{\bf H}_1 \ket{\Psi_2} = \bra{\Psi_2} \hat{\bf H}_2 \ket{\Psi_2} + \bra{\Psi_2} \hat{\bf H}_1-\hat{\bf H}_2 \ket{\Psi_2} \\
    &=  E_{0}^{(2)} + \int_{\R^3} d{\bf x} \blanky n_{0}({\bf x}) [\hat{V}^{(1)}_{\textnormal{ext}}({\bf x})-\hat{V}^{(2)}_{\textnormal{ext}}({\bf x})]
\end{align}

where $E_{0}^{(1)}$ and $E_{0}^{(2)}$ are the groundstates energies of $\hat{\bf H}^{(1)}$ and $\hat{\bf H}^{(2)}$ respectively. It's at this point that the Hohenberg-Kohn theorems, and therefore DFT, apply rigorously to the groundstate only. Note that the previous expression holds as well when the subscripts are interchanged. Thus leading to $ E_{0}^{(1)} +  E_{0}^{(2)} <  E_{0}^{(2)} +  E_{0}^{(1)}$, which is a contradiction, and as a result the groundstate density uniquely determines the external potential (upto an additive constant). \\
\end{proof}

Stated simply, the electrons determine the positions of the nuclei in a system, and also all groundstate electronic properties. Therefore, the external potential and $N$ completely determine the Hamiltonian. \\

The second Hohenberg-Kohn theorem states 

\begin{theo}
\end{theo}
The groundstate energy can be obtained variationally: the density that minimises the total energy is the exact groundstate density. 

\begin{proof}

According to the first Hohenberg-Kohn theorem, $n({\bf x})$ determines $\hat{V}_{\textnormal{ext}}$. $N$ and $\hat{V}_{\textnormal{ext}}$ completely determine $\hat{\bf H}$ and therefore $\Psi$. Therefore, $\Psi$ is a functional of the density, which implies that the expectation value of the electronic Hamiltonian, $\hat{\bf F}$, is a functional of the density as well: $\hat{\bf F} [n({\bf x})] = \bra{\Psi}\hat{\bf F}\ket{\Psi}$. \\

A density that is the groundstate of some external potential is known as $v$-representable. Then, a $v-$representable energy functional $E_{v}[n({\bf x})]$ can be defined in which the external potential $v({\bf x})$ is unrelated to another density $n'({\bf x})$,


\begin{align*}
    E_{v}[n'({\bf x})] = \int_{\R^3} d{\bf x} \blanky n'({\bf x}) v({\bf x}) + \hat{\bf F} [n'({\bf x})],
\end{align*}

then the variational principle asserts,

\begin{equation*}
    \bra{\psi'}\hat{\bf F}\ket{\psi'} + \bra{\psi'}\hat{V}_{\textnormal{ext}}\ket{\psi'} >  \bra{\psi}\hat{\bf F}\ket{\psi} + \bra{\psi}\hat{V}_{\textnormal{ext}}\ket{\psi},
\end{equation*}

where $\psi$ is the wave-function associated with the correct groundstate $n({\bf x})$. This leads to, 

\begin{align*}
     \int_{\R^3} d{\bf x} \blanky n'({\bf x})\hat{V}_{\textnormal{ext}} + \hat{\bf F} [n'({\bf x})] > \int_{\R^3} d{\bf x} \blanky n({\bf x})\hat{V}_{\textnormal{ext}} + \hat{\bf F} [n({\bf x})] \\
     \Rightarrow  E_{v}[n'({\bf x})] >  E_{v}[n({\bf x})],
\end{align*}

which proves the second Hohenberg-Kohn theorem. 

\end{proof}

These two theorems are extremely powerful, though by themselves, do not offer a way of computing the groundstate density in practice. \\

\subsection{Insights into the Hohenberg-Kohn theorems}

Every observable in the groundstate is a unique functional of the electron density. In this section, we'll explore some of the immediate consequences of the previous theorems. \\

Consider the total energy, a measurable quantity with the associated observable being the Hamiltonian ie. 

\begin{equation}
    E[n] = \bra{\Psi} \hat{\bf T}+\hat{\bf V}_{ee}+\hat{V}_{\textnormal{ext}}\ket{\Psi} \rightarrow  E[n] = \bra{\Psi} \hat{\bf H}\ket{\Psi}.
    \label{total energy functional}
\end{equation}

In particular, for the groundstate state total energy, the variational theorem holds, stating that 

\begin{equation}
    \min_{\Psi \in \mathds{H}} \bra{\Psi} \hat{\bf H}\ket{\Psi} =  \bra{\Psi_0} \hat{\bf H}\ket{\Psi_0} = E_0.
\end{equation}

And, given that there is a one-to-one correspondence between a wave-function and a density, the groundstate density corresponds to the groundstate wave-function. Therefore, this minimization can be done directly on the density itself 

\begin{equation}
\min_{n({\bf x})} E[n] = E_0,
\label{variational dft}    
\end{equation}

which is a direct consequence of the second Hohenberg-Kohn theorem. \eqref{variational dft} shows a practical way, via a variational procedure, to find an observable as a density functional. Indeed, rather than using the variational theorem in the realm of all the possible wave functions that live in a $3N$-dimensional space for which the energy surface can be enormously complicated, we can restrict our search to the realm of the densities that live in a smaller-dimensional space, thanks to the Hohenberg-Kohn theorems. \\

For a given density profile, call it $n_1$, there's a unique certain total energy $E[n_1]$. For another density, $n_2$, there's a different total energy $E[n_2]$, which is also unique. This is true in general: $\{n_\lambda\}_{\lambda \in \Lambda} \rightarrow \{E[n_\lambda]\}_{\lambda \in \Lambda}$. Therefore, we can find the minimum of the total energy, which constitutes the groundstate total energy, in correspondence with the groundstate density $n_0$. In practice, this is not the way in which most DFT calculations are performed, but it gives, in principle, a methodology for numerically calculating it. However, there's an unknown relationship between the density and the total energy since we don't know the functional. Formally, the total energy can be written as \eqref{total energy functional}, which can be re-written as 

\begin{align}
    E[n] = \bra{\Psi} \hat{\bf T}+\hat{\bf V}_{ee}\ket{\Psi} + \int_{\R^3} d{\bf x} \blanky \hat{V}_{\textnormal{ext}}({\bf x}) n({\bf x}) \\
    = \hat{\bf T}[n] + \hat{\bf V}_{ee}[n] + \int_{\R^3} d{\bf x} \blanky \hat{V}_{\textnormal{ext}}({\bf x}) n({\bf x}) \\
    = F_{HK}[n] + + \int_{\R^3} d{\bf x} \blanky \hat{V}_{\textnormal{ext}}({\bf x}) n({\bf x})
\end{align}

where, in the second line, the first and second terms are both unknown and are independent of the external potential (its sum yields the so-called Hohenberg-Kohn functional), while the third term gives an explicit functional of the density related to the external potential, since it's a one-body operator. Note that the Hohenberg-Kohn functional is universal for all $N$-electrons systems. It's important to note that the Hohenberg-Kohn's theorems are, in essence, uniqueness theorems: the functional of the density, if it exists, then it's unique. The existence of this functional is related to the idea of $v$-representability. \\

A heuristic idea for $v$-representability can be given as follows: consider a "reasonable" density $n$, reasonable in this context meaning: a positively defined density, normalized to the number of electrons present in the system. Then, this begs the question if it is possible to find an external potential that leads to this density, or in other words, given a density profile, is there a system which has this density?
This question was answered by the works of Levy (1982), Lieb (1982) and Englisch \& Englisch (1983), leading to some novel ideas like 

\begin{itemize}
    \item ensemble-$v$-representable, 
    \item a generalization of the Hohenberg-Kohn theorem, which accounts for degenerate groundstates,
    \item and to generalize $v$-representability to $N$-representability. \\
\end{itemize}

\end{document}
