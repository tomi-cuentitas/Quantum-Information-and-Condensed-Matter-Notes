\documentclass{homework}
\author{Tomás Pérez}
\class{Density Functional Theory - Lecture Notes}
\date{\today}
\title{Theory \& Notes}

\graphicspath{{./media/}}

\begin{document} \maketitle

\section{Quantum Many-Body Problem}

In this section, we're interested in the non-relativistic quantum many body problem, where the system of study is made up of electrons and nuclei. The theory's Hamiltonian is given by

\begin{align}
    \hat{\bf H} = -\frac{\hbar^2}{2m_e} \sum_i \nabla^2_i - \sum_{i, I} \frac{Z_I e^2}{|{\bf x}_i - {\bf R}_I|} + \frac{1}{2} \sum_{i \neq j} \frac{e^2}{|{\bf x}_i-{\bf x}_j|} - \sum_{I} \frac{\hbar^2}{2M_I} \nabla^2_I + \frac{1}{2} \sum_{I \neq J} \frac{Z_I Z_J e^2}{|{\bf R}_I - {\bf R}_J|},
\label{many-body Hamiltonian}
\end{align}

where 

\begin{itemize}
    \item the first corresponds to the sum of the kinetic energy of each one of the electrons,
    \item the second term correspond to the Coulomb interaction created by the nuclei, where the position of $i$-th electron is ${\bf x}_i$ and the nuclei's position is ${\bf R}_I$,
    \item the third term is the Coulomb interaction between the electrons, where we exclude any auto-interaction of an electron with itself. Note that this term forces to consider more than one electron at a time. 
    \item The fourth term is the kinetic energy of the nuclei,
    \item and where the last term accounts for the nucleus-nucleus interaction. \\
\end{itemize}

Solving the general $n$-body problem consists then in finding the eigenfunctions and eigenvalues of the coupled, second order, differential equation given by the action of \eqref{many-body Hamiltonian} on the $n$-body wave function. However, by inspecting the terms of the Hamiltonian we can simplify the problem. Note that the mass of the nuclei is much larger than the mass of the electrons, therefore we can neglect their kinetic energy given by \eqref{many-body Hamiltonian}'s fourth term. Likewise, if we consider the position of the nuclei as parameters, the fifth term adds just a constant. Taking the previous points into consideration, yields a more simplified Hamiltonian:

\begin{equation}
    \hat{\bf H} = -\frac{\hbar^2}{2m_e} \sum_i \nabla^2_i - \sum_{i, I} \frac{Z_I e^2}{|{\bf x}_i - {\bf R}_I|} + \frac{1}{2} \sum_{i \neq j} \frac{e^2}{|{\bf x}_i-{\bf x}_j|} + \sum_{i} {\bf V}_i ({\bf x}_i).
\label{simplified many-body Hamiltonian}
\end{equation}

Note that in said equation, in the second term, we considered only the Coulomb interaction between the electrons and the nuclei. We might as well add any other external potentials acting on the electrons, including the ones stemming from the nuclei, which are accounted for in the fourth term. \\

This is still a highly-non trivial problem to solve. Indeed, solving said problem consists of solving the eigenvalue equation:

\begin{equation}
\hat{\bf H}({\bf x}_1, \cdots, {\bf x}_N) \Psi_\lambda(\tilde{x}_1, \cdots, \tilde{x}_N) = E_\lambda \Psi_\lambda(\tilde{x}_1, \cdots, \tilde{x}_N),
\label{ev equation}
\end{equation}

where 

\begin{itemize}
    \item $\tilde{x}_i = ({\bf x}_i, \sigma_i)$, is an atlas on a Riemannian manifold $V = \R^{3N} \otimes (\mathds{C}^2)^{\otimes N}$, accounting for both the $i$-th electron's position and spin,
    \item and where $\Psi_\lambda$, describing the wavefunction of the $N$-body system, is a function in a Hilbert space given by $\mathds{H} \approx L^2(\R^{3N}) \otimes \bigotimes_{j=1}^{N} \mathds{C}^2 \approx L^2(\R^{3N}) \otimes (\mathds{C}^{2})^{\otimes N}$\footnote{Remember that the Hilbert space for a particle in $\R^3$ with spin is the tensor product of $L^2(\R^3)$ with a finite-dimensional vector space $V$, where $V$ carries and irreducible action of the group rotation $SO(3)$. In this context, the proper notion of "action" is the projective representation of $SO(3)$, meaning a family of operators satisfying the SO(3)-Lie algebra, $\mathfrak{so}(3)$. In our context, the full Hilbert space of a system of $N$ spin-$s$ particles is isomorphic to the tensor product of $N$-Hilbert spaces $\mathds{C}^2$. This may be written in short as $(\mathds{C}^{2s+1})^{\otimes N}$}. Even if we ignore the spatial degrees of freedom, the spin-Hilbert space's dimension scales up as $2^N$. Thus, this problem becomes an NP-hard problem. This is even worse given that we cannot decouple the coordinates of the $i$-th electron from the $j$-th electron. \\ 
\end{itemize}

One way to simplify \eqref{ev equation} is by disregarding the Coulomb interaction and the Pauli principle, considering a system of non-interacting particles. This turns said eigenvalue equation into, a priori, $N$-single electron uncoupled problems, ie.

$$
\Psi_\lambda(\tilde{x}_1, \cdots, \tilde{x}_N) = \prod_{i=1}^{N} \psi_{n_i}(\tilde{x}_i), \textnormal{with } \psi_{n_i}(\tilde{x}) \in L^2(\R^{3}) \otimes \mathds{C}^2 \textnormal{ and } \lambda = \{n_i\}_{i=1}^{N}.
$$

This problem, while tractable, is unphysical since it implies that all electrons move independently in the external potential, allowing for multiple electrons to be in the fundamental state (or any other state for that matter), which goes against the Pauli principle. One way in which we can take into account the Pauli Principle is by taking the Slater determinant of the $N$ single-electron wave functions. Even so, the electrons would still be independent. \\

\section{Observables}

The general purpose for solving the eigenvalue equation \eqref{ev equation} is to then use said wave-function to calculate observables. Consider the grand canonical ensemble and a given operator $\hat{\bf O} : \mathds{H} \rightarrow \mathds{H}$, then 

\begin{equation}
\langle \hat{\bf O} \rangle = \frac{1}{\mathcal{Z}} \sum_{\alpha \in \Lambda} e^{-\beta(E_\alpha - \mu N_\alpha)} \bra{\Psi_\alpha}\hat{\bf O}\ket{\Psi_\alpha},
\label{grand canonical}
\end{equation}

where 

\begin{itemize}
    \item $\Lambda$ indexes the different states,
    \item where $\beta = \frac{1}{kT}$ and $\mu$ is the chemical potential,
    \item $E_\alpha$ and $N_\alpha$ are the energy and particle number in state $\ket{\Psi_\alpha}$ respectively,
    \item and where $\mathcal{Z} =  \sum_{\alpha \in \Lambda} e^{-\beta(E_{\alpha} - \mu N_{\alpha})}$ is the partition function. 
\end{itemize}

In other words, \eqref{grand canonical}, the operator $\hat{\bf O}$ stands for the measurement and is integrated with all the many-body states $\ket{\Psi_\alpha} \in \mathds{H} \approx L^2(\R^{3N}) \otimes (\mathds{C}^{2})^{\otimes N}$ which describe the system. \\

At $N$ fixed and a zero temperature, \eqref{grand canonical} means:

\begin{equation}
     \langle \hat{\bf O} \rangle = \int_{\mathds{R}^{3N}} \prod_{i=1}^N d{\bf x}_i \prod_{j=1}^{N} \sum_{\sigma_j \in \mathds{C}^2} \Psi^{*}(\tilde{x}_1, \cdots, \tilde{x}_N) \sum_{a,b,\cdots} O(x_a, x_b, \cdots) \Psi(\tilde{x}_1, \cdots, \tilde{x}_N),
\label{ensemble avg}
\end{equation}

which can be thought as a functional of the many-body wave function, $\langle \hat{\bf O} \rangle = \langle \hat{\bf O} \rangle[\Psi]: {V}^{\mathds{H}} \rightarrow \R$, with $V = \R^{3N} \times \mathds {C}^{2N}$. Now, note that many details of the many-body wavefunction are integrated out. \\

One way to resolve the many-body problem consists in using an approximate wave-function. For example, Hartree-Fock, configuration interaction and Quantum Monte-Carlo are based in this approach. Another, more radical approach, consists in asking ourselves if we really need to work with all possible many-body wave-functions. \\

Note that \eqref{ensemble avg} depends, at most, with the form of the Hamiltonian, which in our case is given by 
 
\begin{equation}
    \hat{\bf H} = -\frac{\hbar^2}{2m_e} \sum_i \nabla^2_i + \sum_{i} {\bf V}_{\textnormal{ext}} ({\bf x}_i) + \frac{1}{2} \sum_{i \neq j} \frac{e^2}{|{\bf x}_i-{\bf x}_j|}.
\label{hamiltonian 3}
\end{equation}

We consider that the kinetic energy operator and the Coulomb interaction contain enough information to completely determine the value of an observable. Therefore, we can consider the expectation value to be a functional on the external potential, instead of a functional of the many-body wave-function:

$$
\langle \hat{\bf O} \rangle = \langle \hat{\bf O}  \rangle[\Psi] \Rightarrow
\langle \hat{\bf O}  \rangle[V_{\textnormal{ext}}]
: {V}^{\mathds{H}} \rightarrow \R.
$$

Thinking this in terms of functionals, eliminate the explicit dependence on the coordinates $(\tilde{x}_1, \cdots, \tilde{x}_n)$. In order to understand this new functional we note, that in the grand canonical ensemble, the sum of states is actually a trace over the Hilbert space's states, and since the trace is invariant, we can re-write \eqref{ensemble avg} as 

\begin{equation}
    \langle \hat{\bf O} \rangle = \frac{1}{\mathcal{Z}} \Tr_{\mathds{H}} \bigg(e^{-\beta(\hat{\bf H}-\mu\hat{\bf N})} \hat{\bf O} \bigg),
\label{trace gce}
\end{equation}

with the Hamiltonian given by \eqref{hamiltonian 3}. Therefore \eqref{trace gce} depends only on the external potential. Thus, we have expressed a rather complicated functional in terms of a simple function, which we can then try to approximate.

\section{Subsets of interesting observables}

Consider an electronic structure, what are the typical observables of interest? \\

A first example could be the equilibrium distance between two atoms in a molecule, which is the result of an energy-minimization problem. The energy minimization can also give the full geometry of a molecule. 
For example, 

\begin{itemize}
    \item consider a carbon dioxide molecule, the energy minimization problem predicts a linear structure with a flat angle between the two oxygens. Likewise, for the water molecule, the energy minimization problem predicts a specific angle between the hydrogens in said molecule. The same can be said about the ammonia, which is not a planar molecule. 
    \item The crystal structure can also be predicted in this way, by evaluating the forces acting on each atom of the system.
    \item In particular, this method can also compute the lattice vibrations that give rise to phonons or the bulk modulus, the ability of a certain material to resist to a certain external pressure. \\
\end{itemize}

These are very specific examples in which the core of problem is to evaluate the ground-state energy. In DFT, even the excited states of a system are observables as well as some response functions, such as optical absorption or electron energy loss. \\

\section{Observables in terms of Compact Quantities}

This section's aim is to find a simple quantity, in terms of which we can express the observables. Let $(\mathcal{B}(\mathds{H}), ||\cdot||)$ denote the space of all linear operators acting on the Hilbert space, noting that the sub-space of all linear bounded operators is a Banach space. Let $\mathcal{B}_1(\mathds{H})=\{\hat{\bf O}| \hat{\bf O}:\mathds{H}_k \rightarrow \mathds{H}_k \}$ ie. the space of all one-body operators, then we define the space of $k$-body operators as 

$$
\mathcal{B}_k(\mathds{H}) = \{\otimes_{i=1}^{k} {\bf O}_i | {\bf O}_i \in \mathcal{B}_1(\mathds{H}) \},
$$

where, obviously,

$$
\mathcal{B}(\mathds{H}) = \bigsqcup_{i=1}^{N} \mathcal{B}_i(\mathds{H}).
$$

Now, consider a one-body operator $\hat{\bf Q}_i \in \mathcal{B}_{1}(\mathds{H})$, which acts on the $i$-th electron's Hilbert space $\mathds{H}_i$, which we desire to calculate its expected value following equation \eqref{ensemble avg}. Therefore, we can construct a new operator 
$$
\hat{\bf Q} = \bigoplus_{i=1}^{N} \hat{\bf Q}_i = \sum_{i=1}^{N} \tilde{\bf Q}_i
$$

where now both $\hat{\bf Q}$ and $\tilde{\bf Q}_i$ act on the full Hilbert space. Note that $\tilde{\bf Q}_i$ acts trivially on all but one of the individual Hilbert spaces:

\begin{align}
     \langle \hat{\bf Q} \rangle &= \int_{\mathds{R}^{3N}} \prod_{i=1}^N d{\bf x}_i \prod_{j=1}^{N} \sum_{\sigma_j \in \mathds{C}^2} \Psi^{*}(\tilde{x}_1, \cdots, \tilde{x}_N)
     \sum_{i}
     \tilde{\bf Q}_i(\tilde{x_i})
     \Psi(\tilde{x}_1, \cdots, \tilde{x}_N) \\
     &= N \int_{\mathds{R}^{3N}} \prod_{i=1}^N d{\bf x}_i \sum_{\sigma_1 \in \mathds{C}^2} \Psi^{*}(\tilde{x}_1, \cdots, \tilde{x}_N)
     \tilde{\bf Q}_1(\tilde{x}_1)
     \Psi(\tilde{x}_1, \cdots, \tilde{x}_N) \\
     &= \int_{\mathds{R}^{3}} d{\bf x}_1 \sum_{\sigma_1 \in \mathds{C}^2} \tilde{\bf Q}_1(\tilde{x}_1) N \int_{\mathds{R}^{3(N-1)}} \prod_{i=2}^N d{\bf x}_i \prod_{j=2}^{N} \sum_{\sigma_j \in \mathds{C}^2} \Psi^{*}(\tilde{x}_1, \cdots, \tilde{x}_N)
     \Psi(\tilde{x}_1, \cdots, \tilde{x}_N) \\
     &= \int_{\mathds{R}^{3}} d{\bf x}_1 \sum_{\sigma_1 \in \mathds{C}^2} \tilde{\bf Q}_1(\tilde{x}_1) {\bf n}(\tilde{x}_1),
\label{one-body ensemble avg}
\end{align}

where 
 
\begin{equation}
{\bf n}(\tilde{x}_1) = N \int_{\mathds{R}^{3(N-1)}} \prod_{i=2}^N d{\bf x}_i \prod_{j=2}^{N} \sum_{\sigma_j \in \mathds{C}^2} \Psi^{*}(\tilde{x}_1, \cdots, \tilde{x}_N)
     \Psi(\tilde{x}_1, \cdots, \tilde{x}_N),
\label{one-body density}
\end{equation}

is the one-body density. Thus, in order to calculate the expectation value of a one-body operator, we need only the one-body density. Similarly, if we want to compute the expectation value of a general two-body operator, the same steps can be repeated. Consider, for example, the Coulomb interaction. Then it's expectation value is 

\begin{align}
     V_{ee} &= \int_{\mathds{R}^{3N}} \prod_{i=1}^N d{\bf x}_i \prod_{j=1}^{N} \sum_{\sigma_j \in \mathds{C}^2} \Psi^{*}(\tilde{x}_1, \cdots, \tilde{x}_N)
     \sum_{\substack{i,j\\
          j > i}} 
     \frac{e^2}{|{\bf x}_i-{\bf x}_j|}
     \Psi(\tilde{x}_1, \cdots, \tilde{x}_N) \\
     &= \int_{\mathds{R}^{6}} d{\bf x}_1 d{\bf x}_2 \sum_{\sigma_1 \in \mathds{C}^2} \sum_{\sigma_2 \in \mathds{C}^2} \frac{1}{2} \frac{e^2}{|{\bf x}_1-{\bf x}_2|} N(N-1) \int_{\mathds{R}^{3(N-2)}} \prod_{i=3}^N d{\bf x}_i \prod_{j=3}^{N} \sum_{\sigma_j \in \mathds{C}^2} \Psi^{*}(\tilde{x}_1, \cdots, \tilde{x}_N)
     \Psi(\tilde{x}_1, \cdots, \tilde{x}_N) \\
     &=\int_{\mathds{R}^{6}} d{\bf x}_1 d{\bf x}_2 \sum_{\sigma_1 \in \mathds{C}^2} \sum_{\sigma_2 \in \mathds{C}^2} \frac{1}{2} \frac{e^2}{|{\bf x}_1-{\bf x}_2|} {\bf n}^{(2)}(\tilde{x}_1, \tilde{x}_2),
\label{two-body ensemble avg}
\end{align}

where

$$
{\bf n}^{(2)}(\tilde{x}_1, \tilde{x}_2) = N(N-1) \int_{\mathds{R}^{3(N-2)}} \prod_{i=3}^N d{\bf x}_i \prod_{j=3}^{N} \sum_{\sigma_j \in \mathds{C}^2} \Psi^{*}(\tilde{x}_1, \cdots, \tilde{x}_N)
     \Psi(\tilde{x}_1, \cdots, \tilde{x}_N),
$$

is the two-body density.  \\

One interesting observable may be the kinetic  energy, which can readily be written as 

\begin{align}
     T &= -\int_{\mathds{R}^{3N}} \prod_{i=1}^N d{\bf x}_i \prod_{j=1}^{N} \sum_{\sigma_j \in \mathds{C}^2} \Psi^{*}(\tilde{x}_1, \cdots, \tilde{x}_N)
     \sum_{i}\frac{\nabla^{2}_{i}}{2}
     \Psi(\tilde{x}_1, \cdots, \tilde{x}_N) \\
     &= - \int_{\mathds{R}^{3}} d{\bf x}_1 \sum_{\sigma_1 \in \mathds{C}^2} \bigg[\frac{\nabla^{'2}_1}{2} N \int_{\mathds{R}^{3(N-1)}} \prod_{i=2}^N d{\bf x}_i \prod_{j=2}^{N} \sum_{\sigma_j \in \mathds{C}^2} \Psi^{*}(\tilde{x}_1, \cdots, \tilde{x}_N) \Psi(\tilde{x}'_1, \cdots, \tilde{x}_N)\bigg]_{{x'_1 \rightarrow x_1}} \\
     &= \int_{\mathds{R}^{3}} d{\bf x}_1 \sum_{\sigma_1 \in \mathds{C}^2} \frac{\nabla^{'2}_1}{2} {\rho}(\tilde{x}_1, \tilde{x}'_1)_{{x'_1 \rightarrow x_1}},
\end{align}

where, once again, $\rho$ is the one-density matrix 

$$
{\rho}(\tilde{x}_1, \tilde{x}'_1) =  N \int_{\mathds{R}^{3(N-1)}} \prod_{i=2}^N d{\bf x}_i \prod_{j=2}^{N} \sum_{\sigma_j \in \mathds{C}^2} \Psi^{*}(\tilde{x}_1, \cdots, \tilde{x}_N) \Psi(\tilde{x}'_1, \cdots, \tilde{x}_N).
$$

For example, in outlook spectroscopy, other useful ingredients may be the observables associated with transitions to excited states, ie.

\begin{align}
f_{s}^{eh}(\tilde{x}_1, t) &= \int_{\mathds{R}^{3(N-1)}} \prod_{i=2}^N d{\bf x}_i \prod_{j=2}^{N} \sum_{\sigma_j \in \mathds{C}^2} \Psi^{*}(\tilde{x}_1, \cdots, \tilde{x}_N; t) \Psi_{s}(\tilde{x}_1, \cdots, \tilde{x}_N; t) \\
&= e^{-i(E_s-E_0)t} \int_{\mathds{R}^{3(N-1)}} \prod_{i=2}^N d{\bf x}_i \prod_{j=2}^{N} \sum_{\sigma_j \in \mathds{C}^2} \Psi^{*}(\tilde{x}_1, \cdots, \tilde{x}_N) \Psi_{s}(\tilde{x}_1, \cdots, \tilde{x}_N) 
\end{align}

said transition depending on a given number $s$, which indexes the excited state to which the electron has transitioned and where a phase factor has appeared on the second line, an oscillation on the excitation energy. \\

Up to now, in order to calculate the expected value of different $k$-body operators, we need different $N-k$-body kernels, to which integrate them with. Therefore, we first need to calculate said $N-k$-body kernels. However, we insist on finding only one kernel for all different $K-$body operators. \\

%Consider first one-body operators, then the expected values are only functionals of the one-body density given by \eqref{one-body density}.

\section{Calculus of Variations}

A functional is a mapping $F[\cdot]: C^{\infty}(\mathds{R})\rightarrow\mathds{R}$. Integrals, for example, can be thought as functionals, values of a function at a certain point $x_0 \in \mathds{R}$ can also be thought as a functional, with a weight function $W(x) = \delta(x-x_0)$. Some other interesting functionals are, for example, 

\begin{itemize}
    \item Local functional: $F[f] = \int_{\mathds{R}} dx g(f(x))$,
    \item Non-local functionals: 
    $F[f_1, f_2] =\int_{\mathds{R}^2} dx dx' W(x,x') f_1(x) f_2(x')$, 
    \item the total energy of a many-body system 
    $$
    E = F[\Psi] = \int_{\mathds{R}^{3N}} \prod_{i=1}^{N} d{\bf x}_1 \Psi^{*}({\bf x}_1, \cdots, {\bf x}_N) {\bf H} \Psi({\bf x}_1, \cdots, {\bf x}_N) = \bra{\Psi}{\bf H}\ket{\Psi}.
    $$
    \item The Lagrangian action:
    $$
    S[{\bf q}] = \int dt \mathcal{L}({\bf q}(t), \dot{\bf q}(t), t).
    $$
\end{itemize}

In the context of the many-body problem, some interesting functionals of the electron density, could be 

\begin{itemize}
    \item The contribution of the external energy given by an external potential
    
    $$
    E_{\textnormal{ext}}[n] = \int_{\mathds{R}^3} d{\bf x} n({\bf x}) V_{\textnormal{ext}}({\bf x}),
    $$
    
    which is a local functional.
    \item The classical electrostatic term, also called the Hartree energy, is a non-local functional:
    
    $$
    E_{\textnormal{H}}[n] = \frac{1}{2} \int_{\mathds{R}^6} d{\bf x} d{\bf x}' \frac{n({\bf x})n({\bf x'})}{|{\bf x}-{\bf x}'|}.
    $$
    
    \item Another interesting functional is the Thomas-Fermi approximation, in which the kinetic energy is a $5/3$-power of the electron density
    
    $$
    T^{TF}[n] = C \int_{\mathds{R}^3} d{\bf x} n^{\frac{5}{3}}({\bf x}),
    $$
    
    \item or the von-Weizs\"acker functional:
    
    $$
    T^{vW}[n] = \frac{1}{8} \int_{\mathds{R}^3} d{\bf x} \frac{\nabla n({\bf x}) \cdot \nabla n({\bf x})}{n({\bf x})}.
    $$
    
\end{itemize}

\end{document}
