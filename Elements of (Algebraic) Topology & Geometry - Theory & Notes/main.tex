\documentclass{homework}
\author{Tomás Pérez}
\class{Differential Geometry and Topology -Lecture Notes}
\date{\today}
\title{Theory \& Notes}

\graphicspath{{./media/}}

\begin{document} \maketitle

\section{Some topology definitions and results}

First, consider $X$ to be a topological space.\\

\paragraph{\textbf{Path-connected}}

A \textbf{\underline{path}} from a point $p$ to a point $q$ in a topological space $X$ is a continuous function $f: \mathds{R}_{[0,1]} \to X$, with $f(0) = x$ and $f(1) = y$. A \textbf{\underline{path-component}} of $X$ is an equivalence class of $X$ under the equivalence relation which makes $x$ equivalent to $y$ is there there is a path from $x$ to $y$. Hence, $X$ is said to be \textbf{\underline{path-connected}} if there is exactly one-path component, ie. if there is a path joining any two points $X$. Some important results regarding path-connectedness are enunciated, as follows.

\begin{itemize}
    \item Let $X$ and $Y$ be topological spaces and let $f: X \to Y$ be a continuous function. If $X$ is path-connected then its image $f(X)$ is path-connected as well. 
    \item Every path-connected space is connected
    \item The closure of a connected subset is connected. Furthermore, any subset between a connected subset and its closure is connected. 
    \item Every product of a family of path-connected  spaces is path-connected.
    \item Every manifold is locally path-connected. \\
\end{itemize}

\paragraph{\textbf{Simply connected}}

$X$ is \textbf{\underline{simply connected}} (or 1-connected) if it is path-connected and every path between two points can be continuously transformed (intuitively for embedded spaces, staying within the space) into any other such path, while preserving the two endpoints in question. The fundamental group $\pi_1(X)$ is an indicator of the failure of a topological space to be simply connected: a path-connected topological space is simply connected if and only if $\pi_1(X) \simeq \mathds{Z}$. \\

An equivalent definition is: $X$ is called simply-connected if

\begin{itemize}
    \item it is path-connected, 
    \item and any loop in $X$ defined by $f: S^{1} \to X$ can be contracted to a point; ie. $\exists F: D^{2} \to X \textnormal{ such that } F\bigg|_{S^{1}} = f$, where ${S^{1}}$ denotes the unit circle and $D^2$ denotes the closed unit disk in the Euclidean plane respectively. In other words, there exists a continuous map from the closed unit disk to $X$ such that $F$ restricted to $S^1$ is precisely $f$.
\end{itemize}

An equivalent formulation is this: $X$ is simply connected if and only if 

\begin{itemize}
    \item it is path-connected,
    \item and whenever two arbitrary paths (ie. continuous maps) $p: \mathds{R}_{[0,1]} \to X$ and $q: \mathds{R}_{[0,1]} \to X$ with the same endpoints, $p(0) = q(0) \textnormal{ and } p(1) = q(1)$, can be continuously deformed into one another. Explicitly, there exists a homotopy $F: \mathds{R}_{[0,1]} \to X$ such that $F(x,0) = p(x)$ and $F(x,1) = q(x)$. \\
\end{itemize}

Equivalently, $X$ is simply connected if and only if $X$ is path-connected and $\pi_1 (X) \simeq \mathds{Z}$ at each point. \\

\paragraph{\textbf{Homotopy invariants}}

Formally, a homotopy between two continuous functions $f, g: X \to Y$, wherein both $X$ and $Y$ are topological spaces, is defined to be a continuous function $H: X \times \mathds{R}_{[0,1]} \to Y$, from the product of the space $X$ with the unit interval to $Y$, such that 

$$
    H(x,0) = f(x) \textnormal{ and } H(x, 1) = g(x), \forall x \in X.
$$

In particular, a pointed map $f: (X, x_0) \rightarrow (Y, y_0)$ is called \underline{null-homotopic relative to the basepoint} if there is a homotopy $H: X\times \mathds{R}_{[0,1]}$ such that $H(x,0) = f(x)$ and $H(x, 1) = e_{y_0}, \blanky \forall x \in X$, with $f(x_0, t) = y_0$.

In practical terms, if $H$'s second argument is thought of as time, when $H$ describes a continuous deformation of $f$ into $g$, at time 0 we have the function $f$ and at time 1 we have the function $g$, in such a way that there is a smooth transition from $f$ to $g$. \\

Given two topological spaces $X$ and $Y$, a homotopy equivalence between $X$ and $Y$ is a pair of continuous maps $f: X \to Y$ and $g: Y \to X$, such that $ g \circ f $ is homotopic to the identity map on $X$, $\textnormal{id}_X$ and $ f \circ g $ is homotopic to the identity map on $Y$, $\textnormal{id}_Y$. If such a pair exists, then $X$ and $Y$ are homotopy-equivalent, o r of the same homotopy type. Intuitively, two space X and Y are homotopy equivalent if they can be transformed into one another by bending, shrinking and expanding operations. Spaces that are homotopy-equivalent to a point are called contractible.

Note that a homeomorphism is a special case of a homotopy equivalence, in which $g \circ f$ is exactly equal to the identity map $\textnormal{id}_X$ (not only homotopic to it), and $f \circ g$ is equal to $\textnormal{id}_Y$. Therefore, if $X$ and $Y$ are homeomorphic then they are homotopy-equivalent, but the opposite is not true. Some examples:

\begin{itemize}
    \item A solid disk is homotopy-equivalent to a single point, since you can deform the disk along radial lines continuously to a single point. However, they are not homeomorphic, since there is no bijection between them (since one is an infinite set, while the other is finite).
    \item The Möbius strip and an untwisted (closed) strip are homotopy equivalent, since you can deform both strips continuously to a circle. But they are not homeomorphic. \\
\end{itemize}

A function $f$ is said to be null-homotopic if it is homotopic to a constant function. (The homotopy from $f$ to a constant function is then sometimes called a null-homotopy.) For example, a map $f$ from the unit circle $S^1$ to any space $X$ is null-homotopic precisely when it can be continuously extended to a map from the unit disk $D^2$ to $X$ that agrees with $f$ on the boundary. \underline{It follows from these definitions that a space $X$ is contractible if and only if the}
\underline{ identity map from $X$ to itself—which is always a homotopy equivalence—is null-homotopic.} \\

In algebraic topology, there are many interesting homotopic invariants. Namely, let $X, Y$ be topological spaces, then 

\begin{itemize}
    \item $X$ is path-connected if and only if $Y$ is.
    \item $X$ is simply connected if and only if $Y$ is.
    \item if $X$ and $Y$ are path-connected, then the fundamental groups of $X$ and $Y$ are isomorphic, and so are the higher homotopy groups. \\
\end{itemize}

\paragraph{\textbf{Contractibility}}

 Then, said topological space $X$ is \underline{\textbf{contractible}} if the identity map on $X$ is null-homotopic, ie. if it is homotopic to some constant map. Intuitively, a contractible space is one that can be continously shrunk to a point within that space.

A contractible space is precisely one with the homotopy type of a point. It follows that all the the associated homotopy groups of a contractible space are trivial. Therefore, any space with a non-trivial homotopy group cannot be contractible. There are two non-equivalent definitions for a contractible space:

\begin{itemize}
    \item Munkres' definition: Let $X$ be a topological space. If the identity map on $X$, $\textnormal{id}_X$, is null-homotopic, then $X$ is contractible. 
    \item Theodore's definition: Let $X$ be a topological space and let $x_0 \in X$ a point. If there is a continuous map $F: X \times \mathds{R}_{[0,1]} \rightarrow X$ such that 
    
    \begin{align}
    \begin{array}{c}
         F(x, 0) = x \\
         F(x, 1) = x_0 \\
         F(x_0, t) = x_0  
    \end{array} \begin{array}{c}
        \forall x \in X, \\
        \forall t \in \mathds{R}_{[0,1]}
    \end{array} 
    \Rightarrow \textnormal{ then $X$ is contractible. }
    \end{align}
\end{itemize}

For a topological space $X$, the following statements are all equivalent:

\begin{itemize}
    \item $X$ is contractible (ie. the identity map is null-homotopic). 
    \item $X$ is homotopy-equivalent to a one-point space. 
    \item For any space $Y$, any two maps $f, g: Y \rightarrow X$ are homotopic.
    \item For any space $Y$, any map $f: Y \rightarrow X$ is null-homotopic. 
    \item The cone on a space $X$ is always contractible. Therefore any space can be embedded in a contractible one (which also illustrates that subspaces of contractible spaces need not be contractible themselves).
    \item $X$ is a contractibe if and only if there exists a retraction from the cone of $X$ to $X$.
    \item Every contractible space is path-connected and simply connected. Moreover, since all the higher homotopy groups, vanish. Every contractible space is $n$-connected, for all $n \geq 0$.\\ 
\end{itemize}

\paragraph{\textbf{Hopf-Rinow theorem}}

Let $(M,g)$ be a connected Riemannian manifold\footnote{Note that a path-connected topological space is connected as well. 
\begin{tcolorbox}[title=Proof: path-connectedness implies connectedness]
In effect, if $X$ is a path-connected topological space and assuming (by contradiction) that $X = A \sqcup B$, and $A, B \neq \emptyset$ (ie. $X$ isn't connected). Then, let $a \in A, b \in B$ and let $\gamma: \mathds{R}_{[0,1]}$ be a path in $X$ such that $\gamma(0) = a$, $\gamma(1) = b$. Then, given $\gamma$'s continuity, there is a decomposition 

$$
    \mathds{R}_{[0,1]} = \gamma^{-1}(A) \cup \gamma^{-1}(B),
$$

which, if true, implies that $\mathds{R}_{[0,1]}$ is not connected, which is false. Hence, $X$ cannot be path-connected if it isn't connected.
\end{tcolorbox}}. Then, the following statements are equivalent

\begin{itemize}
    \item The closed and bounded subsets of $M$ are compact,
    \item $M$ is a complete metric space, 
    \item $M$ is geodesically complete, that is, $\forall p \in M$, the exponential map $\mathfrak{e}\mathfrak{x}\mathfrak{p}_{p}$ is defined on the entire tangent space $T_p M$,
\end{itemize}

Furthermore, any one of the above implies that, given any two points $p, q \in M$, there exists a length-minimizing geodesic connecting these two points. \\

\section{Preliminary Definitions and Results in Group Theory}

\paragraph{Quotient Group}

Consider the following theorem, concerning groups and quotient groups. These results can be generalized to quotient sets and quotient spaces as well. 

\begin{theorem} \label{Theorem_Quotient_Group}
      Let $G, G'$ be two groups. Then, let $\phi: G \rightarrow G'$ be a homeomorphism of kernel $\phi^{-1}(e') = H \subset G$. Then, $H$ is an invariant subgroup of $G$ and the quotient group is isomorphic to $G'$, 
      $$
        \frac{G}{H} \simeq G'
      $$
\end{theorem}

For further references, see \cite{HoracioII}

\section{Preliminary Definitions and Results in Differential Geometry}

\section{Preliminary Definitions and Results in Point-Set Topology and Algebaic Topology}

\paragraph{Quotient space}

\begin{df}

Let $(X, \tau_X)$ be a topological space with $\tau_X$-topology. If $A$ is a subset of $X$, let $\sim_A$ be the equivalence relation on $X$ defined by 

$$
    x \sim_A y \leftrightarrow x=y \textnormal{ or } (x \in A, y \in A).
$$

This is well-defined, even if $A$ is the empty set. Now, the quotient space $\frac{X}{A}$ is, by definition, $\frac{X}{\sim_A}$. Let $0$ denote the equivalence class $[A]$ of $\sim_A$. Then, as a set,

\begin{equation*}
    \frac{X}{\sim_A} = \left\{
         \begin{array}{cc}
              X & \textnormal{ if } A = \emptyset \\
              (X-A) \cup \{0\} & \textnormal{ if } A \neq \emptyset
         \end{array}
    \right.
\end{equation*}

\end{df}

The open subsets of $\frac{X}{\sim_A}$ are the open subsets of $X-A$ in $X$ and the subsets $U \cup \{0\}$ s.t. $U \cup A$ is open in $X$.

\clearpage

\section{Fiber Bundles}

\clearpage

\section{Geometry and cohomology of K\"ahler manifolds}

\subsection{\textswab{Complex Structure on a vector space}}

Let $\vectorspace$ be a $2m$-dimensional real-valued vector space. A \textit{complex structure} on $\vectorspace$ is an automorphism $J : \vectorspace \rightarrow \vectorspace$ such that $J^2 = - \idop_\vectorspace$. With this structure, $\vectorspace$ is naturally brought into an $m$-dimensional complex-valued vector space by letting 

\begin{equation}
    (\alpha + i\beta) v = \alpha v + \beta J v, \quad \begin{array}{cc}
         v \in \vectorspace  \\
         \alpha, \beta \in \R.
    \end{array}
\end{equation}

In other words, an $m$-dimensional complex-valued vector space can be thought of as a $2m$-dimensional real-value vector space endowed with the complex structure $J = i\id_{\vectorspace}$. 
Hence, this vector space $\vectorspace$ -equipped with the complex structure $J$- has an \textit{adapted basis}

\begin{equation}
    (v_1, \cdots, v_m, Jv_1, \cdots, J v_m), \quad \textnormal{ s.t. } \quad J = \left(
        \begin{array}{cc}
            0 & \idop_\vectorspace  \\
           - \idop_\vectorspace & 0
        \end{array}
    \right).
\end{equation}

\medbreak

An automorphism $\rho: \vectorspace \rightarrow \vectorspace$ preserves a complex structure $J$ on $\vectorspace$ if and only if it commutes with $J$. Hence, these automorphisms form the commutant $\{J\}' \subset \generallineargroup(2m, \R)$ of $J$. It turns out that there is an explicit mapping $\phi$ for the $\{J\}$-commutant of complex-valued $m$-dimensional matrices and the $\{J\}'$-commutant of real-valued $2m$-dimensional matrices. In effect, note that the commutant $\{J\}' \subset \generallineargroup(2m, \R)$ of $J$ is the image of the group $\generallineargroup(m,\C)$ under the monomorphism $\phi$, whose action is given as follows 

\begin{equation}
    \begin{split}
    \phi: \generallineargroup(m,\C) \rightarrow \generallineargroup(2m,\R), \textnormal{ s.t. }
    \begin{array}{cc}
         M \in \generallineargroup(m,\C), \\
         M \overset{\phi}{ \mapsto} 
        \left(
            \begin{array}{cc}
                \Re M & -\Im M  \\
                \Im M & \Re M
            \end{array}
        \right) \in \generallineargroup(2m, \R).
    \end{array}
    \end{split}
\end{equation}|

By invoking \cref{Theorem_Quotient_Group}, one notices that there is an 
explicit one-to-one correspondence between the complex structures 
on a  $2m$-dimensional real-valued vector space $\vectorspace$ and the elements of the quotient
$\frac{\generallineargroup(2m, \R)}{\generallineargroup(m, 
\C)}$\footnote{Here, a short summary of the main topological properties of the real-valued and complex-valued general linear groups is presented.

\begin{enumerate}
    \item The real-valued $\generallineargroup(m, \R)$ is non-compact. Its maximal compact subgroup is the orthogonal group $O(m)$, while the maximal compact subgroup of $\generallineargroup^+(m, \R)$ is the special orthogonal group $SO(m)$. As for $SO(m)$, the group  $\generallineargroup^+(m, \R)$ is not simply connected if $m \neq 1$, but rather has a fundamental group
        
        \begin{equation*}
            \pi_1(SO(m)) = \left\{ \begin{array}{cc}
                 \mathds{Z} \textnormal{ for } m = 2 \\
                \mathds{Z}_2 \textnormal{ for } m > 2 
            \end{array}
            \right..
        \end{equation*}
    
    \item The complex-valued $\generallineargroup(m, \C)$ is a connected space. This follows, in part, since the multiplicative group of complex numbers $\C - \{0\}$ is connected as well. The complex-valued general linear group is not compact however, rather its maximal compact subgroup, $U(m)$, is a compact (group/space). As for $U(m)$, the group manifold $\generallineargroup(m, \C)$ is not simply connected but has a fundamental group $\pi \simeq \mathds{Z}$.
\end{enumerate}
}. \medbreak

\begin{remark}
    \textnormal{First consider the topological space }$X = \generallineargroup(2m,\R)$. \textnormal{Then consider a subspace of it, } $A = \generallineargroup(m,\C)$. \textnormal{The homomorphism } $\phi: \generallineargroup(m,\C) \rightarrow \generallineargroup(2m,\R)$, \textnormal{which in reality is an isomorphism, induces an equivalence relationship } $\sim_{\phi}$ s.t. 
    
    $$
        A \sim_{\phi} B \leftrightarrow A = B, \quad A,B \in \generallineargroup(2m,\R).
    $$
    
   \textnormal{Now, the quotient space } $\frac{\generallineargroup(2m,\R)}{\generallineargroup(m,\C)}$ \textnormal{is, by definition, } $\frac{\generallineargroup(2m,\R)}{\sim_{\phi}}$, \textnormal{given by }
    
    \begin{equation}
        \frac{\generallineargroup(2m,\R)}{\sim_{\phi}} = \{\generallineargroup(2m,\R) - \generallineargroup(m,\C)\} \cup \{0_{\vectorspace}\}.
    \end{equation}
\end{remark}

\bigbreak

A complex structure $J$ on $\vectorspace$ generates a complex structure on the dual space to $\vectorspace$, $\vectorspace^{*}$, as follows 

\begin{equation}
    \langle v, J \omega \rangle = \langle Jv, \omega \rangle, \quad \begin{array}{cc}
         v \in \vectorspace \\
         \omega \in \vectorspace^{*}
    \end{array}.
\end{equation}

\begin{df}
    \textnormal{A scalar product $h: \vectorspace \rightarrow \C$ on a real-valued vector space $\vectorspace$ equipped with a complex structure $J$ is called \textit{Hermitian} if it is $J$-invariant, i.e.}

    \begin{equation}
        h(Jv, Jv') = h(v, v'), \quad v,v' \in \vectorspace.
    \end{equation}

\end{df}

\blanky \bigbreak

From this, it follows immediately that $h(Jv, v) = 0, \blanky \forall v \in \vectorspace$. Moreover, $\vectorspace$ admits an adapted basis, which is orthonormal with respect to this $h$-scalar product. Furthermore, one can also define a skew-symmetric bilinear form, which reads

\begin{equation}\label{Complex_Manifolds_skew_symmetric_form}
    \Omega(v,v') \equiv h(Jv, v'),
\end{equation}

on $\vectorspace$, which is $J$-invariant as well. \smallbreak

One may be interested in defining the so-called \textit{complexification} of a real-valued vector space $\vectorspace$ by considering of a morphism between the rings $\R$ and $\C$, as follows,

\begin{df}
    \textnormal{Let $\vectorspace$ be a real-valued vector space. The \textit{complexification} of $\vectorspace$ is defined as the tensor product of $\vectorspace$ with $\C$, thought of as a two-dimensional real veector space, as follows}
        
        \begin{equation*}
            \vectorspace^{\C} = \C \otimes \vectorspace, \textnormal{ s.t. } \begin{array}{cc}
                 \alpha(v \otimes \beta) = v \otimes (\alpha \beta), \quad v \in \vectorspace, \alpha, \beta \in \C. \\
                 \vectorspace^{\C} \simeq \vectorspace \oplus i \vectorspace \rightarrow v = v_1 \otimes 1 + v_2 \otimes i, \quad v_1, v_2 \in \vectorspace.
            \end{array}
        \end{equation*}
    
    \textnormal{Alternatively, one may use the direct sum as the definition of the complexification $\vectorspace^\C$ of $\vectorspace$ in such a way that , i.e.}
    
    $$
        \vectorspace^\C \equiv \vectorspace \oplus \vectorspace,
    $$    
    \textnormal{ where $\vectorspace^\C$ is imbued with a linear complex structure operator $J$ s.t. 
    $J(v,w) \equiv (-w, v)$. This linear complex structure, thus, encodes the operation "multiplication by $i$" in matrix form. 
    }
\end{df}

\begin{remark}
    \textnormal{From its definition, the complexification $\vectorspace^\C$, the following properties and results hold}
    
    \begin{itemize}
        \item \textnormal{Given a real linear transformation $f: \vectorspace \rightarrow \mathcal{W}$, between two real vector spaces, there is a natural complex linear transformation $f^\C$, the complexification of $f$, $f^\C : \vectorspace^\C \rightarrow \mathcal{W}^\C$ given by}
        
        \begin{equation*}
            f^\C (v\otimes z) = f(v) \otimes z, 
        \end{equation*}
        
        \textnormal{ s.t. } 
        
        \begin{enumerate}
             \item $(\idop_\vectorspace)^\C = \idop_{\vectorspace^\C}$,
             \item $(f \circ g) = f^\C \circ g^\C$,
             \item $(f + g) = f^\C + g^\C$,
             \item $(af)^\C = a f^\C, \quad \in \R$.
        \end{enumerate} 
        
        \textnormal{
        The map $f^\C$ commutes with conjugation, and maps the real subspace of $\vectorspace^\C$ with the real subspace of $\mathcal{W}^\C$. Conversely, a complex linear map $g : \vectorspace^\C \rightarrow \mathcal{W}^\C$ is the complexification of a real linear map if and only if it commutes with conjugation. Hence, it follows that 
        }
        
        $$
            \textnormal{Hom}_{\R}(\vectorspace, \mathcal{W})^\C \simeq \textnormal{Hom}_{\R}(\vectorspace^\C, \mathcal{W}^\C),
        $$
        
        \textnormal{
        where $\textnormal{Hom}_{\R}(\vectorspace, \mathcal{W})$ is the space of all real-valued linear maps from $\vectorspace$ to $\mathcal{W}$.
        }
        
        \item \textnormal{
        The dual $\vectorspace^{*}$ of a real-valued vector space $\vectorspace$ is the space $\vectorspace^{*}$ of all real-valued linear maps from $\vectorspace$ to $\R$. 
        The complexification of $\vectorspace^{*}$ can be naturally be thought of as $\textnormal{Hom}_{\R}(\vectorspace, \C)$. This is,
        }
        
        $$
            (\vectorspace^{*})^\C = {\vectorspace^{*}} \otimes \C \simeq \textnormal{Hom}_{\R}(\vectorspace, \C).
        $$
        
        The isomorphism is given by $(\omega_1 \otimes 1 + \omega_2 \otimes i) \leftrightarrow \omega_1 + i\omega_2, \quad \omega_1, \omega_2 \in \vectorspace^{*}$. 
        
        \item Given a real linear map $\phi : \vectorspace \rightarrow \C$, it may be extended by linearity to yield a complex linear map $\phi: \vectorspace^\C \rightarrow \C$ by letting $\phi(v \otimes z) = z \phi(v)$. 
        
        \item Moreover, the previous extension results in an natural isomorphism between the two following structures
        
        \begin{equation*}
            (\vectorspace^{*})^\C \simeq (\vectorspace^\C)^{*}. 
        \end{equation*}
    \end{itemize}
    
\end{remark}

\blanky \bigbreak

Then, the complexification $\vectorspace$ is a $2m$-dimensional complex space. A complex structure $J$ on $\vectorspace$ is naturally extended to $\vectorspace^\C$ by letting $J \circ i = i \circ J$, allowing $\vectorspace^\C$ to be split into a direct sum of two components

\begin{equation} \label{Complex_Manifolds_holo_antiholo_structures}
\begin{split}
    &\vectorspace^\C = \vectorspace^{1,0} \oplus \vectorspace^{0,1}, \textnormal{ where }\begin{array}{cc}
         \textnormal{$\vectorspace^{1,0}$ is the complex holomorphic subspace }: \vectorspace^{1,0} = \{v + i J v, \blanky v \in \vectorspace\}  \\
         \textnormal{$\vectorspace^{0,1}$ is the complex antiholomorphic subspace }: \vectorspace^{0,1} = \{v - i J v, \blanky v \in \vectorspace\}.
    \end{array}
\end{split}
\end{equation}

These are the eigenspaces of $J$ characterized by the eigenvalues $i$ and $-i$ respectively. Furthermore, there is the \textit{antilinear complex conjugate morphism}  

\begin{equation}
    v = v_1 + iv_2 \mapsto \bar{v} = v_1 - i v_2, \quad \bm{\omega} \mapsto \bar{\bm{\omega}}, \begin{array}{cc}
         \bm{\omega} \in \vectorspace^{r,s} \\
         \bar{\bm{\omega}} \in \vectorspace^{s,r}
    \end{array} r ,s = 0,1, \quad \textnormal{ and s.t. } \bar{Jv} = J \bar{v}.
\end{equation}

From the previous remarks, it is clear that the complexification $(\vectorspace^{*})^\C$ of the dual $\vectorspace^{*}$ of $\vectorspace$ is the complex dual of $\vectorspace^\C$. This allows for a dual decomposition to \cref{Complex_Manifolds_holo_antiholo_structures}, as follows 

\begin{equation} \label{Complex_Manifolds_holo_antiholo_dual_structures}
\begin{split}
    &(\vectorspace^\C)^{*} = (\vectorspace^{1,0})^{*} \oplus (\vectorspace^{0,1})^{*}, \textnormal{ where }\begin{array}{cc}
         \textnormal{$(\vectorspace^{1,0})^{*}$ is the subspace }: (\vectorspace^{1,0})^{*} = \{\omega - i J \omega, \blanky \omega \in \vectorspace^{*}\}  \\
         \textnormal{$(\vectorspace^{0,1})^{*}$ is the subspace }: (\vectorspace^{0,1})^{*} = \{\omega + i J \omega, \blanky \omega \in \vectorspace^{*}\}.
    \end{array}
\end{split}
\end{equation}

are the annihilators\footnote{
   The annihilator of a vector subspace $\mathcal{S}$ of a vector space $\vectorspace$ is the set $S^{0} \subset \vectorspace^{*}$ of linear functionals s.t. $f(s) = 0, \quad s \in \mathcal{S}$.
} of ${\vectorspace^{1,0}}$ and ${\vectorspace^{0,1}}$ respectively. They are the eigenspaces of the complex structure $J$ on $(\vectorspace^{*})^\C$ characterized by the eigenvalues $i$ and $-i$, respectively. \bigbreak

Using the preceding remarks and results, a Hermitian scalar product $h$ on $\vectorspace$ can be uniquely extended to a symmetric complex $J$-invariant bilinear fo on $\vectorspace^{\C}$ fulliling the following conditions 

\begin{itemize}
    \item $h(\xoverline{v}, \xoverline{v}') = \xoverline{h(v,v')}, \quad v,v' \in \vectorspace^\C$.
    \item $h(v, \xoverline{v}) > 0, \quad v \in \vectorspace^{\C} - \{{\bf 0}\}$.
    \item $h(v, v') = 0$ if $v,v'$ are simultaneously holomorphic or antiholomorphic. 
\end{itemize}

This complex bilinear form, thus, induces a non-degenerate Hermitian form on $\vectorspace^\C$, as follows 

$$
    \langle v | v' \rangle_{h} \equiv h(v, \bar{v}'), 
$$

s.t. the holomorphic and antiholomorphic spaces are mutually orthogonal. Accordingly, the skew-symmetric form $\Omega$, introduced in \cref{Complex_Manifolds_skew_symmetric_form}, can be extendd to $\vectorspace^C$ so that 

\begin{equation}
    \begin{array}{cc}
          \Omega(\xoverline{v}, \xoverline{v}') = \xoverline{\Omega(v,v')}, \quad v,v' \in \vectorspace^\C\\ 
          \Omega(v,v') = 0, \quad v,v' \in \vectorspace^{r,s},\quad r,s=0,1. 
    \end{array}
\end{equation}

\clearpage

\subsection{Almost-complex manifolds}

Let $Z$ be a $2m$-dimensional smooth real-valued manifold, with coordinate basis $(z^i)_{i=1}^{2m}.$

%\subsection{.}

\bibliography{citations.bib}
\bibliographystyle{apsrev4-1}

\end{document}

